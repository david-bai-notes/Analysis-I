\section{Limits and Continuity}
If we want to differentiate, we need to make sense of a limit, so here goes.
\subsection{Limit of a function}
\begin{definition}
    Suppose $a\in\mathbb C$ and $f:\mathbb C\setminus\{a\}\to\mathbb C$.
    We say $\lim_{z\to a}f(z)=c\in\mathbb C$ if
    $$\forall\epsilon>0,\exists\delta>0,|z-a|<\delta\implies |f(z)-c|<\epsilon$$
    Sometimes we write $f(z)\to c$ as $z\to a$.
\end{definition}
We say $\lim_{z\to a}f(z)$ is undefined if there is no $c$ with $\lim_{z\to a}f(z)=c$.
\begin{proposition}
    If $\lim_{z\to a}f(z)=c$ and $\lim_{z\to a}f(z)=d$, then $c=d$.
\end{proposition}
\begin{proof}
    Trivial.
\end{proof}
\begin{example}
    1. If $f\equiv c$, then $\lim_{z\to a}f(z)=c$ for any $a$, just take for example $\delta=1$.\\
    2. If $f(z)=z$, then $\lim_{z\to a}f(z)=a$, as we can take $\delta=\epsilon$.\\
    3. If $f(z)=1_{\mathbb C\setminus\{0\}}$, then $\lim_{z\to 0}f(z)=1$, taking for example $\delta=1$.\\
    4. If
    $$f(z)=\begin{cases}
        1\text{, if $\operatorname{Re}(z)>0$}\\
        0\text{, if $\operatorname{Re}(z)=0$}\\
        -1\text{, if $\operatorname{Re}(z)<0$}
    \end{cases}$$
    then $\lim_{z\to 0}f(z)$ does not exist.\\
    5. If $f(z)=z^2$, then $\lim_{z\to a}f(z)=a^2=f(a)$ by choosing $\delta=\min\{\epsilon/(2|a|+1),1\}$.
\end{example}
\begin{proposition}
    $\lim_{z\to a}f(z)=c$ if and only if $\forall (z_n)\in\mathbb C$ with $z_n\to a$, we have $f(z_n)\to c$
\end{proposition}
\begin{proof}
    If the limit of the function is $c$, given $\epsilon>0$, we can find $\delta$ such that $|z-a|<\delta\implies |f(z)-c|<\epsilon$.
    Given a sequence $z_n\to a$, we can find $N\in\mathbb N$ such that $\forall n>N,|z_n-a|<\delta$, so using this $N$, $\forall n>N,|z_n-a|<\delta\implies |f(z_n)-c|<\epsilon$, so $f(z_n)\to c$.\\
    Conversely, if $c$ is not the limit of that function, then $\exists\epsilon>0,\forall\delta>0,\exists z\in\mathbb C,|z-a|<\delta,|f(z)-c|>\epsilon$.
    So we choose such $z_n$ for $\delta=1/n$, then $z_n\to a$ but $|f(z_n)-c|>\epsilon$ for any $n$, hence $f(z_n)$ does not tend to $c$.
\end{proof}
\begin{proposition}
    Let $f,g:\mathbb C\setminus\{a\}\to\mathbb C$, if $\lim_{z\to a}f(z)=c$ and $\lim_{z\to a}f(z)=d$, then\\
    1. $\lim_{z\to a}(f(z)+g(z))=c+d$.\\
    2. $\lim_{z\to a}f(z)g(z)=cd$.\\
    3. If $c\neq 0$, then $\lim_{z\to a}1/f(z)=1/c$.
\end{proposition}
\begin{proof}
    Using the preceding proposition and laws of limits for sequences.
\end{proof}
\begin{proposition}
    Suppose $f,g:\mathbb C\setminus\{a\}\to\mathbb R$ such that
    $$\lim_{z\to a}f(z)=c,\lim_{z\to a}g(z)=d$$
    then,\\
    1. If $\forall z,f(z)\ge g(z)$, then $c\ge d$.\\
    2. (Squeeze Rule) Let $h:\mathbb C\setminus\{a\}$ be such that $f\ge h\ge g$.
    Then if $c=d$, we have $\lim_{z\to a}h(z)=c$.
\end{proposition}
\begin{proof}
    Immediate.
\end{proof}
What if $f:A\to\mathbb C$ where $A$ is a subset of $\mathbb C$?
Does it still makes sense?
Does it change the limit?
The answer to the second question if obviously no.
But we can make sense of the limit.
\begin{definition}
    Suppose $A$ is a subset of $\mathbb C$.
    We say $a\in\mathbb C$ is a limit point of $A$, if $\forall\delta>0,\exists z\in A,0<|z-a|<\delta$.
\end{definition}
\begin{example}
    Any $a\in [0,\infty)$ is a limit point of $(0,\infty)$, and $\mathbb Z$ has no limit point.
\end{example}
\begin{definition}
    If $f:A\to\mathbb C$ and $a\in\mathbb C$ is a limit point of $A$, we say $\lim_{z\to a}f(z)=c$ if $\forall\epsilon>0,\exists\delta>0,z\in A,0<|z-a|<\delta\implies |f(z)-c|<\epsilon$.
\end{definition}
Equivalently, $f(z_n)\to c$ for any sequence $(z_n)\in A$ such that $z_n\to a$.
Also, it is also easy to show that the limit is unique.
\begin{definition}
    For $f:\mathbb R\to\mathbb R$, we define
    $$\lim_{x\to a^+}f(x)=\lim_{x\to a}f|_{(a,\infty)}(x),\lim_{x\to a^-}f(x)=\lim_{x\to a}f|_{(-\infty,a)}(x)$$
\end{definition}
\subsection{Continuity}
\begin{definition}
    Suppose $A\subset\mathbb C$, and $f:A\to\mathbb C$.
    We say $f$ is continuous at $a\in A$ if
    $$\forall\epsilon>0,\exists\delta>0,z\in A,|z-a|<\delta\implies |f(z)-f(a)|<\epsilon$$
\end{definition}
\begin{remark}
    $|f(a)-f(a)|=0<\epsilon$, so this condition is equivalent to say $|f(z)-f(z)|<\epsilon$ whenever $0<|z-a|<\delta$.
\end{remark}
\begin{proposition}
    Suppose $f:A\to\mathbb C$ and $a\in A$, then the following conditions are equivalent:\\
    1. $f$ is continuous at $a$.\\
    2. Either $a$ is not a limit point of $A$ or $\lim_{z\to a}f(z)=f(a)$.\\
    3. $f(z_n)\to f(a)$ whenever $z_n\to a$ is a sequence in $A$.
\end{proposition}
\begin{proof}
    Verify from definition.
\end{proof}
\begin{example}
    1. Constant functions are continuous at every point.
\end{example}
\begin{definition}
    We say $f:A\to\mathbb C$ is continuous if $f$ is continuous at every $a\in A$.
\end{definition}
\begin{definition}
    2. The identity function is continous.\\
    3. (non-example) $f:\mathbb R\to\mathbb R$ by $f=1_{\mathbb Q}$. is not continuous at any point.\\
    4. Consider $f:\mathbb R\to\mathbb R$ by
    $$f(x)=\begin{cases}
        1/q\text{, if $x$ is rational and $q=\min\{q\in\mathbb N:\exists p\in\mathbb Z,x=p/q\}$}\\
        0\text{, otherwise}
    \end{cases}$$
    Then $f$ is not continuous at any $x\in\mathbb Q\setminus\{0\}$ but it is continuous at $a\notin\mathbb Q$.\\
    5. $f:\mathbb R_{\ge 0}\to\mathbb R_{\ge 0}$ by $x\mapsto \sqrt{x}$ is continuous.
\end{definition}
\begin{proposition}
    Suppose $f,g:A\to\mathbb C$ for some $A\subset\mathbb C$ and they are both continuous at $a\in A$.
    Then $f+g,fg$ are continuous at $a$.
    And if $f(a)\neq 0$, then $1/f$ is continuous at $a$.
\end{proposition}
\begin{proof}
    Trivial.
\end{proof}
\begin{example}
    Everywhere in $\mathbb C$ we have any constant function is continous, the identity is continuous, hence any linear function is continuous, therefore any polynomial is continuous.
    Also any rational function is continuous at anywhere it is defined.
\end{example}
The function $\exp$ is continuous, but we will not prove it now.
\begin{proposition}
    Let $A,B,C\subset\mathbb C$ and $f:A\to B,g:B\to C$ such that $f$ is continuous at $a$ and $g$ continuous at $f(a)$, then $g\circ f$ is continuous at $a$.
\end{proposition}
\begin{proof}
    Straightforward.
\end{proof}
\begin{example}
    $g:(0,\infty)\to (0,\infty),g(x)=\sqrt{x}$ and $f:(0,\infty)\to(0,\infty),f(x)=x^2+1$, so $x\mapsto (g\circ f)(x)=\sqrt{x^2+1}$ is continuous.
\end{example}
\begin{lemma}
    Suppose $(x_n)$ is a sequence in $[a,b]$ and $x_n\to x\in\mathbb R$, then $x\in [a,b]$.
\end{lemma}
Note that this is false if we replace $[a,b]$ by $(a,b)$.
\begin{proof}
    Obvious.
\end{proof}
\begin{theorem}[Intermediate Value Theorem]
    If $f:[a,b]\to\mathbb R$ is continuous, and $f(a)\le0,f(b)\ge0$, then there is some $c\in [a,b]$ such that $f(c)=0$.
\end{theorem}
This is false for $\mathbb Q$.
\begin{remark}
    Note that it does not matter if it is $f(a)\le0,f(b)\ge0$ or $f(b)\le0,f(a)\ge0$.
    Also it suffices to show the case where the inequalities are both strict, so we will assume that in the proof.
\end{remark}
\begin{proof}
    For $n\in\mathbb N$, we define $Q_n=\{a+k(b-a)/n:0\le k\le n\}$ and $P_n\subset Q_n$ such that $\forall x\in P_n,f(x)\le 0$, so $P_n$ is never empty.
    Define $x_n=\max P_n$, then $a\le x_n\le b-(b-a)/n$ and $f(x_n)\le 0$.
    We let $y_n=x_n+(b-a)/n=\min Q_n\setminus P_n\subset [a,b]$, then by maximality of $x_n$ we have $f(y_n)>0$.
    By Bolzano-Weierstrass Theorem, there is a subsequence $x_{n_k}$ that converges to some $c\in [a,b]$ by the preceding lemma, so by continuity of $f$, $f(c)=\lim_{n\to\infty}f(x_n)\le 0$.
    Therefore $y_{n_k}\to c$ as well since $(b-a)/n_k\to 0$, but then $f(c)=\lim_{n\to\infty}f(y_n)\ge 0$.
    So $f(c)=0$.
\end{proof}
\begin{corollary}
    An odd degree polynomial has a root in $\mathbb R$.
\end{corollary}
\begin{proof}
    Immediate.
\end{proof}
\begin{corollary}
    If $f:[a,b]\to\mathbb R$ is continuous, and $f(a)=x,f(b)=y$, then for all $z$ in between $x,y$ there is some $c\in [a,b]$ such that $f(c)=z$.
\end{corollary}
\begin{proof}
    $g:x\mapsto x-z$ is continuous.
    Then apply the preceding theorem.
\end{proof}
\begin{theorem}
    Continuous function on bounded closed interval is bounded and attains its bounds.
\end{theorem}
Again it is not true in general for open intervals.
\begin{proof}
    Suffice to show this for upper bounds.
    Let $f:[a,b]\to\mathbb R$ be continuous.
    Let $A=f([a,b])$.
    Suppose $A$ is not bounded above, then $\forall n\in\mathbb N,\exists x_n\in[a,b],f(x_n)>n$.
    Then choose a convergent subsequence $(x_{n_k})$ of $(x_n)$, then $x_{n_k}\to x\in [a,b]$, but by continuity $f(x)=\lim_{k\to\infty}f(x_k)\ge\lim_{k\to\infty}n_k\to\infty$, contradiction.\\
    Now that $A$ is bounded above, we choose $M=\sup A$, then $\forall n\in\mathbb N$, we can choose $x_n\in [a,b]$ such that $M-1/n<x_n\le M$.
    Choose a convergent subsequence $x_{n_k}\to x\in [a,b]$, then continuity tells us that $f(x)=\lim_{k\to\infty}f(x_{n_k})$ which is bounded by $M-1/n_k,M$, hence $f(x)=M$ by Squeeze rule.
\end{proof}
