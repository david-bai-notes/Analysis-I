\section{Convergence of Sequences and Series}
\subsection{Sequence}
\begin{definition}
    Let $X$ be a set, then a sequence in $X$ is a function $f:\mathbb N\to X$.
\end{definition}
We denote the sequence in the form $a_n=f(n)$.
\begin{example}
    1. The constant sequence $f\equiv a$ for some constant $a\in X$.\\
    2. The sequence $f(n)=\sqrt{n^2+17}$ is a sequence in positive real numbers.\\
    3. Flip a coin.
    The sequence
    $$a_n=\begin{cases}
        1\text{, if the $n^{th}$ flip is head}\\
        0\text{, if it is tail.}
    \end{cases}$$
    Then $a_n$ is a sequence in $\{0,1\}$.\\
    4. With last example, we can define $b_1=0,b_{n+1}=b_n+a_n$ makes a sequence $b_n$ in $\mathbb Z$.
\end{example}
\begin{definition}
    If $(z_n)$ is a sequence of complex numbers, we say $z_n\to z$ (or $z_n$ converges to $z$) as $n\to\infty$ if
    $$\forall\epsilon>0,\exists N\in\mathbb N,\forall n>N,|z_n-z|<\epsilon$$
    Otherwise, we say $(z_n)$ diverges.
\end{definition}
Note that $N$ depends on $\epsilon$.
\begin{example}
    1. The constant sequence $z_n=z$ converges to $z$.\\
    2. The sequence $z_n=1/n$ converges to $0$.
    Given $\epsilon>0$ we can find $N$ such that $N>1/\epsilon$ since $\mathbb N$ has no upper bound, then whenever $n>N$, we have $|z_n-z|=|1/n|<1/N<\epsilon$.\\
    3. The sequence $z_n=n$ diverges.\\
    4. The sequence $z_n=(-1)^n$ diverges.
\end{example}
\begin{proposition}
    $z_n\to z$ iff:\\
    1. $z_n-z\to 0$.\\
    2. $|z_n-z|\to 0$.\\
    3. $\forall m\in N,\exists N\in\mathbb N, \forall n>N,|z_n-z|<1/m$.
\end{proposition}
\begin{proposition}
    Suppose $z_n$ and $w_n$ are sequences in $\mathbb C$.
    If $z_n\to z,w_n\to w$, then\\
    1. $z_n+w_n\to z+w$.\\
    2. $z_nw_n\to zw$.\\
    3. If $z\neq 0$, then $\exists N\in\mathbb N,\forall n>N,z_n\neq 0$ and for $n>N,1/z_n\to 1/z$.\\
    4. If $z_n=x_n+iy_n,z=x+iy$ with $x_n,x,y_n,y\in\mathbb R$, we have $z_n\to z\iff x_n\to x\land y_n\to y$.
\end{proposition}
\begin{proof}
    Trivial.
\end{proof}
\begin{example}
    1. $1/(n^2)=(1/n)(1/n)\to 0\cdot 0=0$.\\
    2. $1\pm1/n^2\to 1\pm0=1$.\\
    3. $1/(1\pm 1/n^2)\to 1/1=1$.\\
    4. $(n^2-n)/(n^2+n)=(1+1/n)^{-1}(1-1/n)\to 1\cdot 1=1$.
\end{example}
\begin{remark}
    2 implies that if $z_n\to z$ then $cz_n\to cz$.
    Similarly 2,3 together show that if $z_n\to z,w_n\to w\neq 0$, then $z_n/w_n$ is eventually well-defined and goes to $z/w$.
\end{remark}
\begin{corollary}[Uniqueness of Limit]
    If $z_n\to z$ and $z_n\to z'$, then $z=z'$
\end{corollary}
\begin{proof}
    By 1 and 2, $0\equiv z_n+(-1)z_n\to z+(-1)z'=z-z'$, but if $0\to w\neq 0$, then $|w|>0$, therefore there is some $n$ such that $|w|=|w-0|<|w|/2<|w|$, contradiction.
    So $z=z'$.
\end{proof}
We sometimes write
$$\lim_{n\to\infty}z_n=z$$
if $z_n\to z$.
We can write it so since limits are unique.
\begin{proposition}
    Suppose $x_n\to x,y_n\to y$ are real sequences, then if $\forall n,x_n\ge y_n$, then $x>y$.
\end{proposition}
Note that it is not true if we replace $\ge$ by $>$.
\begin{proof}
    It suffices to prove the case when $y_n\equiv 0$.\\
    If $x<0$, then there is a natural number $n$ such that $|x_n-x|<|x|$, but $|x_n-x|\ge|x|$, which is a contradiction.\\
    For the general case, just consider the sequence $x_n-y_n$.
\end{proof}
\begin{proposition}[Squeeze Rule]
    If $x_n,c_n,y_n$ are real sequences with $\forall n,x_n\ge c_n\ge y_n$ and $x_n\to c,y_n\to c$, then $c_n\to c$.
\end{proposition}
\begin{proof}
    Given $\epsilon>0$, pick $N_1$ such that $\forall n>N_1,|x_n-c|<\epsilon$ and $N_2$ such that $\forall n>N_2,|y_n-c|<\epsilon$, so $\forall n>\max\{N_1,N_2\}$. we have
    $$c+\epsilon>x_n\ge c_n\ge y_n>c-\epsilon$$
    so $|c_n-c|<\epsilon$.
\end{proof}
\begin{example}
    Since
    $$\frac{n^2-1}{n^2}\le\frac{n^2+\sin n}{n^2}\le\frac{n^2+1}{n^2}$$
    we have $(n^2+\sin n)/n^2\to 1$.
\end{example}
\begin{definition}
    Let $x_n$ be a sequence in $\mathbb R$.
    We say it is monotone increasing if $x_{n+1}\ge x_n$ for all $n\mathbb N$.
\end{definition}
\begin{definition}
    A sequence $x_n$ is bounded above if there is some constant $c\in\mathbb R$ such that $x_n\ge c$ for any $n$.
\end{definition}
\begin{theorem}[Monotone Sequence Theorem]
    A monotone increasing sequence in $\mathbb R$ that is bounded above converges.
\end{theorem}
Note that this theorem is false if we replace $\mathbb R$ by $\mathbb Q$
\begin{proof}
    Let $x_n$ be such a sequence and we let $c=\sup_{n\in\mathbb N}x_n$.
    $c$ exists since $x_n$ is bounded above.
    We shall show that $x_n\to c$.\\
    If $x_n$ does not tend to $c$, then there is some $\epsilon>0$ such that $\forall N\in\mathbb N,\exists n>N,c-x_n=|x_n-c|\ge\epsilon$.
    However, this would mean that for any $N\in\mathbb N$, we can choose such an $n$, then it gives $x_N\le x_n\le c-\epsilon$, so $c$ is not the least upper bound, contradiction.
\end{proof}
\begin{definition}
    Let $(x_n)$ be a sequence in $\mathbb R$, we say $x_n\to\infty$ if $\forall M\in\mathbb R,\exists N\in\mathbb N,\forall n>N,x_n>M$.
\end{definition}
Note that a sequence that tends to infinity diverges.
\begin{proposition}
    If the sequence $(x_n)$ is monotone increasing, then it either converges or tends to infinity.
\end{proposition}
\subsection{Series}
\begin{definition}
    We say the infinite series
    $$\sum_{i=0}^\infty c_i$$
    converges (diverges) if the sequence $s_n=\sum_{i=0}^nc_i$ converges (diverges).\\
    If $s_n$ converges to $s$, we write
    $$\sum_{i=0}^\infty c_i=s$$
\end{definition}
\begin{example}
    Consider the geometric series $\sum_{k=0}^\infty z^k$ where $|z|<1$.
    So $s_n=(1-z^{n+1})/(1-z)$.
    Note that $z_{n+1}\to 0$ as $n\to\infty$, hence $s_n\to 1/(1-z)$.
\end{example}
\begin{lemma}
    If the series sum $\sum_{i=0}^\infty c_i$ converges, then $c_i\to 0$ as $i\to\infty$.
\end{lemma}
Note that the converse of the statement is false.
\begin{proof}
    Trivial.
\end{proof}
\begin{lemma}
    Suppose $(a_n),(b_n)$ are sequences in $\mathbb C$,\\
    1. For any $K_1,K_2\in\mathbb N$, $\sum_{i=K_1}^\infty a_n$ cconverges iff $\sum_{i=K_2}^\infty a_n$ converges.\\
    2. Suppose $\sum_na_n\to a,\sum_nb_n\to b$, then $\sum_n(a_n+\lambda b_n)\to a+\lambda b$.
\end{lemma}
\begin{proof}
    Obvious.
\end{proof}
\begin{theorem}[Comparison Test]
    Suppose $(a_n),(b_n)$ are sequences in $\mathbb R$, and $a_n\ge b_n\ge 0$ for all $n$, then if the series $\sum_ia_i$ converges, so does $\sum_ib_i$, and $\sum_ia_i\ge\sum_ib_i\ge 0$.
\end{theorem}
\begin{proof}
    Indubitable.
    \footnote{The author is short of such sort of adjectives.}
\end{proof}
\begin{definition}
    Let $(c_n)$ be a complex sequence, we say $\sum_ic_i$ converges absolutely if $\sum_i|c_i|$ converges.
\end{definition}
\begin{lemma}
    If $(c_n)\in\mathbb R$, then if $\sum_{i}|c_i|$ converges, then $\sum_ic_i$ converges.
\end{lemma}
\begin{proof}
    Just sum up $c_n+|c_n|$ and use Comparison Test.
\end{proof}
\begin{theorem}
    If $(c_n)\in\mathbb C$ converges absolutely, then it converges.
\end{theorem}
\begin{proof}
    Split real and imaginary part and use Comparison Test.
\end{proof}
Alternatively one can also use triangle inequality.
\begin{theorem}[Strong Comparison Test]
    Suppose $(c_n)\in\mathbb C$ and $(a_n)\in\mathbb R_+$ with $a_n\ge |c_n|\ge 0$.
    Then if $\sum_na_n$ converges, then $\sum_nc_n$ converges absolutely.
\end{theorem}
\begin{proof}
    Analogous.
\end{proof}
\begin{theorem}[Ratio Test]
    Suppose $(c_n)\in\mathbb C$ such that
    $$[0,\infty]\ni r=\lim_{n\to\infty}\left|\frac{c_{n+1}}{c_n}\right|$$
    exists, then if $r<1$, $\sum_{n}c_n$ converges, and if $r>1$, $\sum_nc_n$ diverges.
\end{theorem}
If $r=1$, then both ways are possible.
\begin{example}
    $$\sum_{n=0}^\infty\frac{z^n}{n!}$$
    where $z$ is a complex number.
    This series converges absolutely since
    $$\lim_{n=\to\infty}\frac{|z|^{n+1}n!}{|z|^n(n+1)!}=\lim_{n\to\infty} \frac{|z|}{n+1}\to 0<1$$
    by Ratio Test.
    So we can really define $\exp(z)$.
\end{example}
\begin{proof}
    If $r>1$, then we can find $\eta$ such that $1<\eta<r$, and $\exists N\in\mathbb N,\forall n>N$ we have $|c_{n+1}|>\eta|c_n|$, therefore $|c_n|\to\infty$, hence the series diverges.\\
    If $r<1$, then we can find $\eta$ such that $r<\eta<1$ and $\exists N\in\mathbb N,\forall n>N,|c_{n+1}|\le\eta|c_n|$, so a comparison test with the geometric series shows the convergence. 
\end{proof}
\begin{example}[Non-examples]
    Consider the harmonic series $\sum_n1/n$.
    The ratio test is useless here as $(n+1)/n\to 1$ as $n\to\infty$.
    The sequence of the terms converges as well, so we cannot get anything from there either.
    And in fact, it diverges.
    We can compare it with the sequence $1+1/2+1/4+1/4+1/8+1/8+1/8+1/8+\cdots$ to show that it is indeed actually diverges.
\end{example}
\begin{proposition}[Cauchy Condensation Test]
    Suppose $(a_n)\in\mathbb R$ and $a_n\ge a_{n+1}\ge 0$ for any $n$.
    Then $\sum_na_n$ converges iff $\sum_{k}2^ka_{2^k}$ converges.
\end{proposition}
I'd like to think that we can replace $2$ by any natural number $p\ge 2$.
\begin{example}
    $\sum_nn^{-p}$ converges iff $\sum_k2^k2^{-kp}=\sum_k(2^{1-p})^k$ converges iff $p>1$.
\end{example}
\begin{proof}
    Comparison.
\end{proof}
\begin{definition}
    Suppose $(x_n)$ is a sequence in some set $X$.
    A subsequence of $x_n$ is a sequence of the form $x_{n_k}$ where $(n_k)_{k\in\mathbb N}$ is strictly increasing.
\end{definition}
\begin{lemma}
    Suppose $c_n$ is a sequence in $\mathbb C$ and $c_{n_k}$ is a subsequence, then if $c_n\to c$, then $c_{n_k}\to c$
\end{lemma}
\begin{proof}
    Immediate.
\end{proof}
The converse is obviously not true.
\begin{lemma}
    If $(a_n)\in\mathbb R$ is monotone and a subsequence of it converges $a_{n_k}\to a$, then $a_n\to a$.
\end{lemma}
\begin{proof}
    Easy.
\end{proof}
\subsection{Bolzano-Weierstrass}
\begin{definition}
    Let $(c_n)$ be a sequence of complex numbers.
    We say $(c_n)$ is bounded if $\exists M>0,\forall n\in\mathbb N,|c_n|<M$.
\end{definition}
\begin{theorem}[Bolzano-Weierstrass]
    Any bounded sequence has a converging subsequence.
\end{theorem}
This is false in $\mathbb Q$ as one will expect.\\
We shall prove this by ``lion-hunting''.
The philosophy is like this:
Given an infinite sequence of lions in a rectangular zone.
We divide the rectangle by halving each side, then one of the four parts shall contain infinitely many lions.
Do this again on the region having infinitely many lions.
And do this again and again and again so there is a sequence of rectangles $A_0\subset A_1\subset A_2\subset\cdots$ such that each $A_n$ contains infinitely many lions and that the diameters of $A_n$ decreases (exponentially) to $0$ when $n\to\infty$.
So choosing a lion in each $A_n$ (such that its index is higher than the previous lion, possible since each $A_n$ is infinite) gives a subsequence of converging lions.
\begin{proof}
    Suffices to show the real case.
    Take $a_0=-M,b_0=M$, then $[a_0,b_0]$ contains infinitely many terms of the sequence.
    Once we have chosen $a_n,b_n$ such that $[a_n,b_n]$ contains infinitely many terms of the sequence, at least one of $[a_n,(a_n+b_n)/2],[(a_n+b_n)/2,b_n]$ must contain infinitely many terms of the sequence.
    For the former case we set $a_{n+1}=a_n,b_{n+1}=(a_n+b_n)/2$ and for the latter $a_{n+1}=(a_n+b_n)/2,b_{n+1}=b_n$.
    Then $b_{n+1}-a_{n+1}=(b_n-a_n)/2=2M/2^{n+1}\to 0$ as $n\to\infty$.
    Also both $b_n,a_n$ converges as monotone sequences.
    Then we can choose $x_{n_k}$ inductively:
    Choose $x_{n_0}\in [a_0,b_0]$, then once we have chosen $x_{n_k}$, we can choose $n_{k+1}$ by choosing one such that $x_{n_{k+1}}\in [a_{k+1},b_{k+1}]$ and $n_{k+1}>n_k$, which is possible as $[a_{k+1},b_{k+1}]$ contains infinitely many terms.
    So $x_{n_k}$ is squeezed by $a_k,b_k$ which converges to the same limit, hence this subsequence converges.
\end{proof}
\begin{definition}
    Let $(z_n)\in\mathbb C$ be a sequence in $\mathbb C$, we say $z_n$ is Cauchy if $\forall\epsilon>0,\exists N\in\mathbb N,\forall n,m>N,|x_n-x_m|<\epsilon$.
\end{definition}
\begin{proposition}
    A complex sequence converges if and only if it is Cauchy.
\end{proposition}
\begin{proof}
    The ``only if'' part is obvious.
    For the other direction, it is easy that Cauchy sequences are bounded and that if it has a subsequence that converges, then it converges to the same value.
    Then Bolzano-Weierstrass suffices.
\end{proof}
\begin{corollary}
    Let $(z_n)\in\mathbb C$, then $\sum_nz_n$ converges iff $\forall\epsilon>0,\exists N\in\mathbb N,\forall n>m>N$,
    $$\left|\sum_{k=m+1}^nz_k\right|<\epsilon$$
\end{corollary}
\begin{proof}
    Immediate.
\end{proof}
\begin{proposition}[Alternating Series Test]
    Suppose $(a_n)\in\mathbb R$ and $a_n\ge a_{n+1}\ge 0$ and $a_n\to 0$, then
    $$\sum_{n=0}^\infty(-1)^na_n$$
    converges.
\end{proposition}
\begin{example}
    So $\sum_n(-1)^n/n$ converges even if the harmonic series diverges.
\end{example}
\begin{proof}
    Let $s_n$ be the $n^{th}$ partial sum, then $s_{2n+1}=s_{2n-1}+a_{2n}-a_{2n+1}\ge s_{2n-1}$ is monotone increasing and bounded above by $a_0$, so it converges.
    Also $s_{2n}$ converges as well and we can squeeze it.
\end{proof}