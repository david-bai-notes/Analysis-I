\section{Introduction}
\begin{theorem}
    Let $F,f:[a,b]\to\mathbb R$ are continuous and $F$ differentiable with $F^\prime=f$, we have
    $$\int_a^bf(t)\,\mathrm dt=F(b)-F(a)$$
\end{theorem}
There is a reason why this theorem takes so long to prove.
First of all obvious we need to actually define what we meant by all those terms in there and link them together.
More subtlely, this is a theorem about the real numbers.
Suppose $F^\prime=f=G^\prime$, then $F$ and $G$ can be differed by a constant.
So if we set
$$F(x)=\begin{cases}
    1\text{, if $x^2>2$}\\
    0\text{, otherwise}
\end{cases}$$
and $G\equiv 0$.
So $F$ is differentiable at every rational number and those derivatives are $0$, and so is $G$, but $F$ is not constant.
Hence the theorem is simply not true for rational numbers, so we must use properties of real numbers.\\
These are what real analysis (in a beginner's level) is going to be about.