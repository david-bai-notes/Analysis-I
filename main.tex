\documentclass[a4paper]{article}

\usepackage{hyperref}

\newcommand{\triposcourse}{Analysis I}
\newcommand{\triposterm}{Lent 2020}
\newcommand{\triposlecturer}{Prof. J. Rasmussen}
\newcommand{\tripospart}{IA}

\usepackage{amsmath}
\usepackage{amssymb}
\usepackage{amsthm}
\usepackage{mathrsfs}

\usepackage{tikz-cd}

\theoremstyle{plain}
\newtheorem{theorem}{Theorem}[section]
\newtheorem{lemma}[theorem]{Lemma}
\newtheorem{proposition}[theorem]{Proposition}
\newtheorem{corollary}[theorem]{Corollary}
\newtheorem{problem}[theorem]{Problem}
\newtheorem*{claim}{Claim}

\theoremstyle{definition}
\newtheorem{definition}{Definition}[section]
\newtheorem{conjecture}{Conjecture}[section]
\newtheorem{example}{Example}[section]

\theoremstyle{remark}
\newtheorem*{remark}{Remark}
\newtheorem*{note}{Note}

\title{\triposcourse{}
\thanks{Based on the lectures under the same name taught by \triposlecturer{} in \triposterm{}.}}
\author{Zhiyuan Bai}
\date{Compiled on \today}

\setcounter{section}{-1}

\begin{document}
    \maketitle
    This document serves as a set of revision materials for the Cambridge Mathematical Tripos Part \tripospart{} course \textit{\triposcourse{}} in \triposterm{}.
    However, despite its primary focus, readers should note that it is NOT a verbatim recall of the lectures, since the author might have made further amendments in the content.
    Therefore, there should always be provisions for errors and typos while this material is being used.
    \tableofcontents
    \section{Introduction}
\begin{theorem}
    Let $F,f:[a,b]\to\mathbb R$ are continuous and $F$ differentiable with $F^\prime=f$, we have
    $$\int_a^bf(t)\,\mathrm dt=F(b)-F(a)$$
\end{theorem}
There is a reason why this theorem takes so long to prove.
First of all obvious we need to actually define what we meant by all those terms in there and link them together.
More subtlely, this is a theorem about the real numbers.
Suppose $F^\prime=f=G^\prime$, then $F$ and $G$ can be differed by a constant.
So if we set
$$F(x)=\begin{cases}
    1\text{, if $x^2>2$}\\
    0\text{, otherwise}
\end{cases}$$
and $G\equiv 0$.
So $F$ is differentiable at every rational number and those derivatives are $0$, and so is $G$, but $F$ is not constant.
Hence the theorem is simply not true for rational numbers, so we must use properties of real numbers.\\
These are what real analysis (in a beginner's level) is going to be about.
    \section{The Real and Complex Numbers}
\begin{definition}
    The real numbers is a set $\mathbb R$ such that\\
    1. The rational numbers $\mathbb Q$ is contained in $\mathbb R$.\\
    2. $\mathbb R$ is an ordered field.\\
    3. It satisfies the least-upper-bound property.
\end{definition}
$\mathbb R$ is a field if it has two binary operations $+,\cdot$ such that $(\mathbb R,+),(\mathbb R\setminus\{0\},\cdot)$ are abelian groups and $\forall a,b,c\in\mathbb R,(a+b)c=ac+bc$.
\begin{example}
    1. $\mathbb Q$ is a field.\\
    2. $\mathbb Z/p\mathbb Z$ is a field for $p$ prime and no otherwise.
\end{example}
An ordered field means a field such that there is a total order $<$ on $\mathbb R$ such that $a>b\implies a+c>b+c$ and $a>0,b>0\implies ab>0$.
\begin{example}
    $\mathbb Q$ is an ordered field but $\mathbb Z/p\mathbb Z$ is not.
\end{example}
\begin{definition}
    Let $X$ be an ordered set and $A\subset X$.
    We say $A$ has an upper bound in $X$ if $\exists a\in X,\forall x\in A,x\le a$.
    We call $a$ an upper bound for $A$.
    If that is true, we say $A$ is bounded above.\\
    We say $a\in X$ is the least upper bound, or supremum of $A$ if it is an upper bound and $a'\ge a$ whenever $a'$ is an upper bound of $A$.
    So we write $a=\sup A$ since it is obvious that supremums are unique.
\end{definition}
\begin{definition}
    We say $X$ has the LUBP if any nonempty subset of $X$ that is bounded above has a supremum.
\end{definition}
Note that $\mathbb Q$ does not have LUBP by considering $\{x\in\mathbb Q:x^2<2\}$.\\
These definitions can be easily extended to lower bounds and greatest lower bounds and greatest-lower-bound property.
We write $\inf$ or infermum for the greatest lower bound.
\begin{proposition}
    1. Given the rational numbers, we can construct the real numbers satisfying the axioms.
    2. If $X$ is an ordered field that satisfies the LUBP, then there is an isomorphism between ordered fields $\phi:X\to\mathbb R$.
\end{proposition}
\begin{proof}
    Partially obvious.
    Some might be covered later.
\end{proof}
\begin{proposition}
    1. The natural numbers is not bounded above.\\
    2. $\forall x\in\mathbb R,\exists N\in\mathbb N,N>x$.\\
    3. $\forall x\in\mathbb R_{>0},\exists N\in\mathbb N,1/N<x$.\\
    4. $\forall x,y\in\mathbb R,x<y\implies\exists q\in\mathbb Q,x<q<y$.\\
    5. $\forall x\in\mathbb R_{>0},\exists y\in\mathbb R,y^2=x$.
\end{proposition}
\begin{proof}
    All obvious.
\end{proof}
\begin{definition}
    The complex numbers $\mathbb C$ is
    $$\left\{ z\in M_{2\times 2}(\mathbb R):z=\begin{pmatrix}
        a&b\\
        -b&a
    \end{pmatrix}\right\}$$
    which becomes a field under matrix addition and multiplication.
    Let $1=I$ and $i=\left(\begin{smallmatrix}
        0&1\\
        -1&0
    \end{smallmatrix}\right)$, then we have an obvious bijection $\mathbb R^2\to\mathbb C$ with $(a,b)\mapsto a+bi$.
\end{definition}
\begin{definition}
    If $z=a+bi$, we set the modulus, or norm, by $|z|=\sqrt{\det z}=\sqrt{a^2+b^2}$.
\end{definition}
Note that $|zw|=|z||w|$.
\begin{proposition}
    $\forall z,w\in\mathbb C,|z+w|\le |z|+|w|$.
\end{proposition}
\begin{proof}
    Trivial.
\end{proof}
    \section{Convergence of Sequences and Series}
\subsection{Sequence}
\begin{definition}
    Let $X$ be a set, then a sequence in $X$ is a function $f:\mathbb N\to X$.
\end{definition}
We denote the sequence in the form $a_n=f(n)$.
\begin{example}
    1. The constant sequence $f\equiv a$ for some constant $a\in X$.\\
    2. The sequence $f(n)=\sqrt{n^2+17}$ is a sequence in positive real numbers.\\
    3. Flip a coin.
    The sequence
    $$a_n=\begin{cases}
        1\text{, if the $n^{th}$ flip is head}\\
        0\text{, if it is tail.}
    \end{cases}$$
    Then $a_n$ is a sequence in $\{0,1\}$.\\
    4. With last example, we can define $b_1=0,b_{n+1}=b_n+a_n$ makes a sequence $b_n$ in $\mathbb Z$.
\end{example}
\begin{definition}
    If $(z_n)$ is a sequence of complex numbers, we say $z_n\to z$ (or $z_n$ converges to $z$) as $n\to\infty$ if
    $$\forall\epsilon>0,\exists N\in\mathbb N,\forall n>N,|z_n-z|<\epsilon$$
    Otherwise, we say $(z_n)$ diverges.
\end{definition}
Note that $N$ depends on $\epsilon$.
\begin{example}
    1. The constant sequence $z_n=z$ converges to $z$.\\
    2. The sequence $z_n=1/n$ converges to $0$.
    Given $\epsilon>0$ we can find $N$ such that $N>1/\epsilon$ since $\mathbb N$ has no upper bound, then whenever $n>N$, we have $|z_n-z|=|1/n|<1/N<\epsilon$.\\
    3. The sequence $z_n=n$ diverges.\\
    4. The sequence $z_n=(-1)^n$ diverges.
\end{example}
\begin{proposition}
    $z_n\to z$ iff:\\
    1. $z_n-z\to 0$.\\
    2. $|z_n-z|\to 0$.\\
    3. $\forall m\in N,\exists N\in\mathbb N, \forall n>N,|z_n-z|<1/m$.
\end{proposition}
\begin{proposition}
    Suppose $z_n$ and $w_n$ are sequences in $\mathbb C$.
    If $z_n\to z,w_n\to w$, then\\
    1. $z_n+w_n\to z+w$.\\
    2. $z_nw_n\to zw$.\\
    3. If $z\neq 0$, then $\exists N\in\mathbb N,\forall n>N,z_n\neq 0$ and for $n>N,1/z_n\to 1/z$.\\
    4. If $z_n=x_n+iy_n,z=x+iy$ with $x_n,x,y_n,y\in\mathbb R$, we have $z_n\to z\iff x_n\to x\land y_n\to y$.
\end{proposition}
\begin{proof}
    Trivial.
\end{proof}
\begin{example}
    1. $1/(n^2)=(1/n)(1/n)\to 0\cdot 0=0$.\\
    2. $1\pm1/n^2\to 1\pm0=1$.\\
    3. $1/(1\pm 1/n^2)\to 1/1=1$.\\
    4. $(n^2-n)/(n^2+n)=(1+1/n)^{-1}(1-1/n)\to 1\cdot 1=1$.
\end{example}
\begin{remark}
    2 implies that if $z_n\to z$ then $cz_n\to cz$.
    Similarly 2,3 together show that if $z_n\to z,w_n\to w\neq 0$, then $z_n/w_n$ is eventually well-defined and goes to $z/w$.
\end{remark}
\begin{corollary}[Uniqueness of Limit]
    If $z_n\to z$ and $z_n\to z'$, then $z=z'$
\end{corollary}
\begin{proof}
    By 1 and 2, $0\equiv z_n+(-1)z_n\to z+(-1)z'=z-z'$, but if $0\to w\neq 0$, then $|w|>0$, therefore there is some $n$ such that $|w|=|w-0|<|w|/2<|w|$, contradiction.
    So $z=z'$.
\end{proof}
We sometimes write
$$\lim_{n\to\infty}z_n=z$$
if $z_n\to z$.
We can write it so since limits are unique.
\begin{proposition}
    Suppose $x_n\to x,y_n\to y$ are real sequences, then if $\forall n,x_n\ge y_n$, then $x>y$.
\end{proposition}
Note that it is not true if we replace $\ge$ by $>$.
\begin{proof}
    It suffices to prove the case when $y_n\equiv 0$.\\
    If $x<0$, then there is a natural number $n$ such that $|x_n-x|<|x|$, but $|x_n-x|\ge|x|$, which is a contradiction.\\
    For the general case, just consider the sequence $x_n-y_n$.
\end{proof}
\begin{proposition}[Squeeze Rule]
    If $x_n,c_n,y_n$ are real sequences with $\forall n,x_n\ge c_n\ge y_n$ and $x_n\to c,y_n\to c$, then $c_n\to c$.
\end{proposition}
\begin{proof}
    Given $\epsilon>0$, pick $N_1$ such that $\forall n>N_1,|x_n-c|<\epsilon$ and $N_2$ such that $\forall n>N_2,|y_n-c|<\epsilon$, so $\forall n>\max\{N_1,N_2\}$. we have
    $$c+\epsilon>x_n\ge c_n\ge y_n>c-\epsilon$$
    so $|c_n-c|<\epsilon$.
\end{proof}
\begin{example}
    Since
    $$\frac{n^2-1}{n^2}\le\frac{n^2+\sin n}{n^2}\le\frac{n^2+1}{n^2}$$
    we have $(n^2+\sin n)/n^2\to 1$.
\end{example}
\begin{definition}
    Let $x_n$ be a sequence in $\mathbb R$.
    We say it is monotone increasing if $x_{n+1}\ge x_n$ for all $n\mathbb N$.
\end{definition}
\begin{definition}
    A sequence $x_n$ is bounded above if there is some constant $c\in\mathbb R$ such that $x_n\ge c$ for any $n$.
\end{definition}
\begin{theorem}[Monotone Sequence Theorem]
    A monotone increasing sequence in $\mathbb R$ that is bounded above converges.
\end{theorem}
Note that this theorem is false if we replace $\mathbb R$ by $\mathbb Q$
\begin{proof}
    Let $x_n$ be such a sequence and we let $c=\sup_{n\in\mathbb N}x_n$.
    $c$ exists since $x_n$ is bounded above.
    We shall show that $x_n\to c$.\\
    If $x_n$ does not tend to $c$, then there is some $\epsilon>0$ such that $\forall N\in\mathbb N,\exists n>N,c-x_n=|x_n-c|\ge\epsilon$.
    However, this would mean that for any $N\in\mathbb N$, we can choose such an $n$, then it gives $x_N\le x_n\le c-\epsilon$, so $c$ is not the least upper bound, contradiction.
\end{proof}
\begin{definition}
    Let $(x_n)$ be a sequence in $\mathbb R$, we say $x_n\to\infty$ if $\forall M\in\mathbb R,\exists N\in\mathbb N,\forall n>N,x_n>M$.
\end{definition}
Note that a sequence that tends to infinity diverges.
\begin{proposition}
    If the sequence $(x_n)$ is monotone increasing, then it either converges or tends to infinity.
\end{proposition}
\subsection{Series}
\begin{definition}
    We say the infinite series
    $$\sum_{i=0}^\infty c_i$$
    converges (diverges) if the sequence $s_n=\sum_{i=0}^nc_i$ converges (diverges).\\
    If $s_n$ converges to $s$, we write
    $$\sum_{i=0}^\infty c_i=s$$
\end{definition}
\begin{example}
    Consider the geometric series $\sum_{k=0}^\infty z^k$ where $|z|<1$.
    So $s_n=(1-z^{n+1})/(1-z)$.
    Note that $z_{n+1}\to 0$ as $n\to\infty$, hence $s_n\to 1/(1-z)$.
\end{example}
\begin{lemma}
    If the series sum $\sum_{i=0}^\infty c_i$ converges, then $c_i\to 0$ as $i\to\infty$.
\end{lemma}
Note that the converse of the statement is false.
\begin{proof}
    Trivial.
\end{proof}
\begin{lemma}
    Suppose $(a_n),(b_n)$ are sequences in $\mathbb C$,\\
    1. For any $K_1,K_2\in\mathbb N$, $\sum_{i=K_1}^\infty a_n$ cconverges iff $\sum_{i=K_2}^\infty a_n$ converges.\\
    2. Suppose $\sum_na_n\to a,\sum_nb_n\to b$, then $\sum_n(a_n+\lambda b_n)\to a+\lambda b$.
\end{lemma}
\begin{proof}
    Obvious.
\end{proof}
\begin{theorem}[Comparison Test]
    Suppose $(a_n),(b_n)$ are sequences in $\mathbb R$, and $a_n\ge b_n\ge 0$ for all $n$, then if the series $\sum_ia_i$ converges, so does $\sum_ib_i$, and $\sum_ia_i\ge\sum_ib_i\ge 0$.
\end{theorem}
\begin{proof}
    Indubitable.
    \footnote{The author is short of such sort of adjectives.}
\end{proof}
\begin{definition}
    Let $(c_n)$ be a complex sequence, we say $\sum_ic_i$ converges absolutely if $\sum_i|c_i|$ converges.
\end{definition}
\begin{lemma}
    If $(c_n)\in\mathbb R$, then if $\sum_{i}|c_i|$ converges, then $\sum_ic_i$ converges.
\end{lemma}
\begin{proof}
    Just sum up $c_n+|c_n|$ and use Comparison Test.
\end{proof}
\begin{theorem}
    If $(c_n)\in\mathbb C$ converges absolutely, then it converges.
\end{theorem}
\begin{proof}
    Split real and imaginary part and use Comparison Test.
\end{proof}
Alternatively one can also use triangle inequality.
\begin{theorem}[Strong Comparison Test]
    Suppose $(c_n)\in\mathbb C$ and $(a_n)\in\mathbb R_+$ with $a_n\ge |c_n|\ge 0$.
    Then if $\sum_na_n$ converges, then $\sum_nc_n$ converges absolutely.
\end{theorem}
\begin{proof}
    Analogous.
\end{proof}
\begin{theorem}[Ratio Test]
    Suppose $(c_n)\in\mathbb C$ such that
    $$[0,\infty]\ni r=\lim_{n\to\infty}\left|\frac{c_{n+1}}{c_n}\right|$$
    exists, then if $r<1$, $\sum_{n}c_n$ converges, and if $r>1$, $\sum_nc_n$ diverges.
\end{theorem}
If $r=1$, then both ways are possible.
\begin{example}
    $$\sum_{n=0}^\infty\frac{z^n}{n!}$$
    where $z$ is a complex number.
    This series converges absolutely since
    $$\lim_{n=\to\infty}\frac{|z|^{n+1}n!}{|z|^n(n+1)!}=\lim_{n\to\infty} \frac{|z|}{n+1}\to 0<1$$
    by Ratio Test.
    So we can really define $\exp(z)$.
\end{example}
\begin{proof}
    If $r>1$, then we can find $\eta$ such that $1<\eta<r$, and $\exists N\in\mathbb N,\forall n>N$ we have $|c_{n+1}|>\eta|c_n|$, therefore $|c_n|\to\infty$, hence the series diverges.\\
    If $r<1$, then we can find $\eta$ such that $r<\eta<1$ and $\exists N\in\mathbb N,\forall n>N,|c_{n+1}|\le\eta|c_n|$, so a comparison test with the geometric series shows the convergence. 
\end{proof}
\begin{example}[Non-examples]
    Consider the harmonic series $\sum_n1/n$.
    The ratio test is useless here as $(n+1)/n\to 1$ as $n\to\infty$.
    The sequence of the terms converges as well, so we cannot get anything from there either.
    And in fact, it diverges.
    We can compare it with the sequence $1+1/2+1/4+1/4+1/8+1/8+1/8+1/8+\cdots$ to show that it is indeed actually diverges.
\end{example}
\begin{proposition}[Cauchy Condensation Test]
    Suppose $(a_n)\in\mathbb R$ and $a_n\ge a_{n+1}\ge 0$ for any $n$.
    Then $\sum_na_n$ converges iff $\sum_{k}2^ka_{2^k}$ converges.
\end{proposition}
I'd like to think that we can replace $2$ by any natural number $p\ge 2$.
\begin{example}
    $\sum_nn^{-p}$ converges iff $\sum_k2^k2^{-kp}=\sum_k(2^{1-p})^k$ converges iff $p>1$.
\end{example}
\begin{proof}
    Comparison.
\end{proof}
\begin{definition}
    Suppose $(x_n)$ is a sequence in some set $X$.
    A subsequence of $x_n$ is a sequence of the form $x_{n_k}$ where $(n_k)_{k\in\mathbb N}$ is strictly increasing.
\end{definition}
\begin{lemma}
    Suppose $c_n$ is a sequence in $\mathbb C$ and $c_{n_k}$ is a subsequence, then if $c_n\to c$, then $c_{n_k}\to c$
\end{lemma}
\begin{proof}
    Immediate.
\end{proof}
The converse is obviously not true.
\begin{lemma}
    If $(a_n)\in\mathbb R$ is monotone and a subsequence of it converges $a_{n_k}\to a$, then $a_n\to a$.
\end{lemma}
\begin{proof}
    Easy.
\end{proof}
\subsection{Bolzano-Weierstrass}
\begin{definition}
    Let $(c_n)$ be a sequence of complex numbers.
    We say $(c_n)$ is bounded if $\exists M>0,\forall n\in\mathbb N,|c_n|<M$.
\end{definition}
\begin{theorem}[Bolzano-Weierstrass]
    Any bounded sequence has a converging subsequence.
\end{theorem}
This is false in $\mathbb Q$ as one will expect.\\
We shall prove this by ``lion-hunting''.
The philosophy is like this:
Given an infinite sequence of lions in a rectangular zone.
We divide the rectangle by halving each side, then one of the four parts shall contain infinitely many lions.
Do this again on the region having infinitely many lions.
And do this again and again and again so there is a sequence of rectangles $A_0\subset A_1\subset A_2\subset\cdots$ such that each $A_n$ contains infinitely many lions and that the diameters of $A_n$ decreases (exponentially) to $0$ when $n\to\infty$.
So choosing a lion in each $A_n$ (such that its index is higher than the previous lion, possible since each $A_n$ is infinite) gives a subsequence of converging lions.
\begin{proof}
    Suffices to show the real case.
    Take $a_0=-M,b_0=M$, then $[a_0,b_0]$ contains infinitely many terms of the sequence.
    Once we have chosen $a_n,b_n$ such that $[a_n,b_n]$ contains infinitely many terms of the sequence, at least one of $[a_n,(a_n+b_n)/2],[(a_n+b_n)/2,b_n]$ must contain infinitely many terms of the sequence.
    For the former case we set $a_{n+1}=a_n,b_{n+1}=(a_n+b_n)/2$ and for the latter $a_{n+1}=(a_n+b_n)/2,b_{n+1}=b_n$.
    Then $b_{n+1}-a_{n+1}=(b_n-a_n)/2=2M/2^{n+1}\to 0$ as $n\to\infty$.
    Also both $b_n,a_n$ converges as monotone sequences.
    Then we can choose $x_{n_k}$ inductively:
    Choose $x_{n_0}\in [a_0,b_0]$, then once we have chosen $x_{n_k}$, we can choose $n_{k+1}$ by choosing one such that $x_{n_{k+1}}\in [a_{k+1},b_{k+1}]$ and $n_{k+1}>n_k$, which is possible as $[a_{k+1},b_{k+1}]$ contains infinitely many terms.
    So $x_{n_k}$ is squeezed by $a_k,b_k$ which converges to the same limit, hence this subsequence converges.
\end{proof}
\begin{definition}
    Let $(z_n)\in\mathbb C$ be a sequence in $\mathbb C$, we say $z_n$ is Cauchy if $\forall\epsilon>0,\exists N\in\mathbb N,\forall n,m>N,|x_n-x_m|<\epsilon$.
\end{definition}
\begin{proposition}
    A complex sequence converges if and only if it is Cauchy.
\end{proposition}
\begin{proof}
    The ``only if'' part is obvious.
    For the other direction, it is easy that Cauchy sequences are bounded and that if it has a subsequence that converges, then it converges to the same value.
    Then Bolzano-Weierstrass suffices.
\end{proof}
\begin{corollary}
    Let $(z_n)\in\mathbb C$, then $\sum_nz_n$ converges iff $\forall\epsilon>0,\exists N\in\mathbb N,\forall n>m>N$,
    $$\left|\sum_{k=m+1}^nz_k\right|<\epsilon$$
\end{corollary}
\begin{proof}
    Immediate.
\end{proof}
\begin{proposition}[Alternating Series Test]
    Suppose $(a_n)\in\mathbb R$ and $a_n\ge a_{n+1}\ge 0$ and $a_n\to 0$, then
    $$\sum_{n=0}^\infty(-1)^na_n$$
    converges.
\end{proposition}
\begin{example}
    So $\sum_n(-1)^n/n$ converges even if the harmonic series diverges.
\end{example}
\begin{proof}
    Let $s_n$ be the $n^{th}$ partial sum, then $s_{2n+1}=s_{2n-1}+a_{2n}-a_{2n+1}\ge s_{2n-1}$ is monotone increasing and bounded above by $a_0$, so it converges.
    Also $s_{2n}$ converges as well and we can squeeze it.
\end{proof}
    \subsection{Bonus: Construct the Reals from the Rationals}
The definition of convergence and Cauchy sequence makes sense in $\mathbb Q$ as well.
The matters, there are sequences in $\mathbb Q$ that ``want to converge'' (being Cauchy) but don't, like $1,1.4,1.41,1.414$.
In fact, every $x\in\mathbb R$ is a limit of a sequence in $\mathbb Q$ (by e.g. decimal expansion).
So we can define
$$\mathbb R=\{(x_n)\in\mathbb Q: (x_n)\text{ is Cauchy}\}/\sim$$
where $\sim$ is an equivalence relation defined by $(x_n)\sim (y_n)\iff x_n-y_n\to 0$.
\begin{proposition}
    This is indeed the real number with the intuitive definition of operations and order.
\end{proposition}
In particular $(x_n)\ge0=(0,0,\ldots)$ iff $\exists N\in\mathbb n,\forall n>N,x_n\ge 0$.
\begin{proof}
    Other properties are trivial, so we shall check the least-upper-bound property.
    Note that the least-upper-bound property, the monotone sequence theorem and the statement that every Cauchy sequence converges are equivalent.
    So we shall prove that every Cauchy sequence converges.
    Suppose $(x_{m,n})=((x_m)_n)$ is a Cauchy sequence in $\mathbb R$ defined the way we made it.
    Then the sequence $(y_k)=x_{k,k}$ goes to the limit we want.
\end{proof}
    \section{Limits and Continuity}
If we want to differentiate, we need to make sense of a limit, so here goes.
\subsection{Limit of a function}
\begin{definition}
    Suppose $a\in\mathbb C$ and $f:\mathbb C\setminus\{a\}\to\mathbb C$.
    We say $\lim_{z\to a}f(z)=c\in\mathbb C$ if
    $$\forall\epsilon>0,\exists\delta>0,|z-a|<\delta\implies |f(z)-c|<\epsilon$$
    Sometimes we write $f(z)\to c$ as $z\to a$.
\end{definition}
We say $\lim_{z\to a}f(z)$ is undefined if there is no $c$ with $\lim_{z\to a}f(z)=c$.
\begin{proposition}
    If $\lim_{z\to a}f(z)=c$ and $\lim_{z\to a}f(z)=d$, then $c=d$.
\end{proposition}
\begin{proof}
    Trivial.
\end{proof}
\begin{example}
    1. If $f\equiv c$, then $\lim_{z\to a}f(z)=c$ for any $a$, just take for example $\delta=1$.\\
    2. If $f(z)=z$, then $\lim_{z\to a}f(z)=a$, as we can take $\delta=\epsilon$.\\
    3. If $f(z)=1_{\mathbb C\setminus\{0\}}$, then $\lim_{z\to 0}f(z)=1$, taking for example $\delta=1$.\\
    4. If
    $$f(z)=\begin{cases}
        1\text{, if $\operatorname{Re}(z)>0$}\\
        0\text{, if $\operatorname{Re}(z)=0$}\\
        -1\text{, if $\operatorname{Re}(z)<0$}
    \end{cases}$$
    then $\lim_{z\to 0}f(z)$ does not exist.\\
    5. If $f(z)=z^2$, then $\lim_{z\to a}f(z)=a^2=f(a)$ by choosing $\delta=\min\{\epsilon/(2|a|+1),1\}$.
\end{example}
\begin{proposition}
    $\lim_{z\to a}f(z)=c$ if and only if $\forall (z_n)\in\mathbb C$ with $z_n\to a$, we have $f(z_n)\to c$
\end{proposition}
\begin{proof}
    If the limit of the function is $c$, given $\epsilon>0$, we can find $\delta$ such that $|z-a|<\delta\implies |f(z)-c|<\epsilon$.
    Given a sequence $z_n\to a$, we can find $N\in\mathbb N$ such that $\forall n>N,|z_n-a|<\delta$, so using this $N$, $\forall n>N,|z_n-a|<\delta\implies |f(z_n)-c|<\epsilon$, so $f(z_n)\to c$.\\
    Conversely, if $c$ is not the limit of that function, then $\exists\epsilon>0,\forall\delta>0,\exists z\in\mathbb C,|z-a|<\delta,|f(z)-c|>\epsilon$.
    So we choose such $z_n$ for $\delta=1/n$, then $z_n\to a$ but $|f(z_n)-c|>\epsilon$ for any $n$, hence $f(z_n)$ does not tend to $c$.
\end{proof}
\begin{proposition}
    Let $f,g:\mathbb C\setminus\{a\}\to\mathbb C$, if $\lim_{z\to a}f(z)=c$ and $\lim_{z\to a}f(z)=d$, then\\
    1. $\lim_{z\to a}(f(z)+g(z))=c+d$.\\
    2. $\lim_{z\to a}f(z)g(z)=cd$.\\
    3. If $c\neq 0$, then $\lim_{z\to a}1/f(z)=1/c$.
\end{proposition}
\begin{proof}
    Using the preceding proposition and laws of limits for sequences.
\end{proof}
\begin{proposition}
    Suppose $f,g:\mathbb C\setminus\{a\}\to\mathbb R$ such that
    $$\lim_{z\to a}f(z)=c,\lim_{z\to a}g(z)=d$$
    then,\\
    1. If $\forall z,f(z)\ge g(z)$, then $c\ge d$.\\
    2. (Squeeze Rule) Let $h:\mathbb C\setminus\{a\}$ be such that $f\ge h\ge g$.
    Then if $c=d$, we have $\lim_{z\to a}h(z)=c$.
\end{proposition}
\begin{proof}
    Immediate.
\end{proof}
What if $f:A\to\mathbb C$ where $A$ is a subset of $\mathbb C$?
Does it still makes sense?
Does it change the limit?
The answer to the second question if obviously no.
But we can make sense of the limit.
\begin{definition}
    Suppose $A$ is a subset of $\mathbb C$.
    We say $a\in\mathbb C$ is a limit point of $A$, if $\forall\delta>0,\exists z\in A,0<|z-a|<\delta$.
\end{definition}
\begin{example}
    Any $a\in [0,\infty)$ is a limit point of $(0,\infty)$, and $\mathbb Z$ has no limit point.
\end{example}
\begin{definition}
    If $f:A\to\mathbb C$ and $a\in\mathbb C$ is a limit point of $A$, we say $\lim_{z\to a}f(z)=c$ if $\forall\epsilon>0,\exists\delta>0,z\in A,0<|z-a|<\delta\implies |f(z)-c|<\epsilon$.
\end{definition}
Equivalently, $f(z_n)\to c$ for any sequence $(z_n)\in A$ such that $z_n\to a$.
Also, it is also easy to show that the limit is unique.
\begin{definition}
    For $f:\mathbb R\to\mathbb R$, we define
    $$\lim_{x\to a^+}f(x)=\lim_{x\to a}f|_{(a,\infty)}(x),\lim_{x\to a^-}f(x)=\lim_{x\to a}f|_{(-\infty,a)}(x)$$
\end{definition}
\subsection{Continuity}
\begin{definition}
    Suppose $A\subset\mathbb C$, and $f:A\to\mathbb C$.
    We say $f$ is continuous at $a\in A$ if
    $$\forall\epsilon>0,\exists\delta>0,z\in A,|z-a|<\delta\implies |f(z)-f(a)|<\epsilon$$
\end{definition}
\begin{remark}
    $|f(a)-f(a)|=0<\epsilon$, so this condition is equivalent to say $|f(z)-f(z)|<\epsilon$ whenever $0<|z-a|<\delta$.
\end{remark}
\begin{proposition}
    Suppose $f:A\to\mathbb C$ and $a\in A$, then the following conditions are equivalent:\\
    1. $f$ is continuous at $a$.\\
    2. Either $a$ is not a limit point of $A$ or $\lim_{z\to a}f(z)=f(a)$.\\
    3. $f(z_n)\to f(a)$ whenever $z_n\to a$ is a sequence in $A$.
\end{proposition}
\begin{proof}
    Verify from definition.
\end{proof}
\begin{example}
    1. Constant functions are continuous at every point.
\end{example}
\begin{definition}
    We say $f:A\to\mathbb C$ is continuous if $f$ is continuous at every $a\in A$.
\end{definition}
\begin{definition}
    2. The identity function is continous.\\
    3. (non-example) $f:\mathbb R\to\mathbb R$ by $f=1_{\mathbb Q}$. is not continuous at any point.\\
    4. Consider $f:\mathbb R\to\mathbb R$ by
    $$f(x)=\begin{cases}
        1/q\text{, if $x$ is rational and $q=\min\{q\in\mathbb N:\exists p\in\mathbb Z,x=p/q\}$}\\
        0\text{, otherwise}
    \end{cases}$$
    Then $f$ is not continuous at any $x\in\mathbb Q\setminus\{0\}$ but it is continuous at $a\notin\mathbb Q$.\\
    5. $f:\mathbb R_{\ge 0}\to\mathbb R_{\ge 0}$ by $x\mapsto \sqrt{x}$ is continuous.
\end{definition}
\begin{proposition}
    Suppose $f,g:A\to\mathbb C$ for some $A\subset\mathbb C$ and they are both continuous at $a\in A$.
    Then $f+g,fg$ are continuous at $a$.
    And if $f(a)\neq 0$, then $1/f$ is continuous at $a$.
\end{proposition}
\begin{proof}
    Trivial.
\end{proof}
\begin{example}
    Everywhere in $\mathbb C$ we have any constant function is continous, the identity is continuous, hence any linear function is continuous, therefore any polynomial is continuous.
    Also any rational function is continuous at anywhere it is defined.
\end{example}
The function $\exp$ is continuous, but we will not prove it now.
\begin{proposition}
    Let $A,B,C\subset\mathbb C$ and $f:A\to B,g:B\to C$ such that $f$ is continuous at $a$ and $g$ continuous at $f(a)$, then $g\circ f$ is continuous at $a$.
\end{proposition}
\begin{proof}
    Straightforward.
\end{proof}
\begin{example}
    $g:(0,\infty)\to (0,\infty),g(x)=\sqrt{x}$ and $f:(0,\infty)\to(0,\infty),f(x)=x^2+1$, so $x\mapsto (g\circ f)(x)=\sqrt{x^2+1}$ is continuous.
\end{example}
\begin{lemma}
    Suppose $(x_n)$ is a sequence in $[a,b]$ and $x_n\to x\in\mathbb R$, then $x\in [a,b]$.
\end{lemma}
Note that this is false if we replace $[a,b]$ by $(a,b)$.
\begin{proof}
    Obvious.
\end{proof}
\begin{theorem}[Intermediate Value Theorem]
    If $f:[a,b]\to\mathbb R$ is continuous, and $f(a)\le0,f(b)\ge0$, then there is some $c\in [a,b]$ such that $f(c)=0$.
\end{theorem}
This is false for $\mathbb Q$.
\begin{remark}
    Note that it does not matter if it is $f(a)\le0,f(b)\ge0$ or $f(b)\le0,f(a)\ge0$.
    Also it suffices to show the case where the inequalities are both strict, so we will assume that in the proof.
\end{remark}
\begin{proof}
    For $n\in\mathbb N$, we define $Q_n=\{a+k(b-a)/n:0\le k\le n\}$ and $P_n\subset Q_n$ such that $\forall x\in P_n,f(x)\le 0$, so $P_n$ is never empty.
    Define $x_n=\max P_n$, then $a\le x_n\le b-(b-a)/n$ and $f(x_n)\le 0$.
    We let $y_n=x_n+(b-a)/n=\min Q_n\setminus P_n\subset [a,b]$, then by maximality of $x_n$ we have $f(y_n)>0$.
    By Bolzano-Weierstrass Theorem, there is a subsequence $x_{n_k}$ that converges to some $c\in [a,b]$ by the preceding lemma, so by continuity of $f$, $f(c)=\lim_{n\to\infty}f(x_n)\le 0$.
    Therefore $y_{n_k}\to c$ as well since $(b-a)/n_k\to 0$, but then $f(c)=\lim_{n\to\infty}f(y_n)\ge 0$.
    So $f(c)=0$.
\end{proof}
\begin{corollary}
    An odd degree polynomial has a root in $\mathbb R$.
\end{corollary}
\begin{proof}
    Immediate.
\end{proof}
\begin{corollary}
    If $f:[a,b]\to\mathbb R$ is continuous, and $f(a)=x,f(b)=y$, then for all $z$ in between $x,y$ there is some $c\in [a,b]$ such that $f(c)=z$.
\end{corollary}
\begin{proof}
    $g:x\mapsto x-z$ is continuous.
    Then apply the preceding theorem.
\end{proof}
\begin{theorem}
    Continuous function on bounded closed interval is bounded and attains its bounds.
\end{theorem}
Again it is not true in general for open intervals.
\begin{proof}
    Suffice to show this for upper bounds.
    Let $f:[a,b]\to\mathbb R$ be continuous.
    Let $A=f([a,b])$.
    Suppose $A$ is not bounded above, then $\forall n\in\mathbb N,\exists x_n\in[a,b],f(x_n)>n$.
    Then choose a convergent subsequence $(x_{n_k})$ of $(x_n)$, then $x_{n_k}\to x\in [a,b]$, but by continuity $f(x)=\lim_{k\to\infty}f(x_k)\ge\lim_{k\to\infty}n_k\to\infty$, contradiction.\\
    Now that $A$ is bounded above, we choose $M=\sup A$, then $\forall n\in\mathbb N$, we can choose $x_n\in [a,b]$ such that $M-1/n<x_n\le M$.
    Choose a convergent subsequence $x_{n_k}\to x\in [a,b]$, then continuity tells us that $f(x)=\lim_{k\to\infty}f(x_{n_k})$ which is bounded by $M-1/n_k,M$, hence $f(x)=M$ by Squeeze rule.
\end{proof}

    \section{Differentiation}
Until further notice, we will just study differentiation of a function $f:I\to\mathbb R$ where $I$ is an interval.
\subsection{The Derivative}
\begin{definition}
    If $a\in I$ and we have
    $$\lim_{x\to a}\frac{f(x)-f(a)}{x-a}$$
    exists and equals $c\in\mathbb R$, we say $f$ is differentiable at $a$ and the derivative is $f^\prime(a)=c$.
    Moreover, if $f$ is differentiable at all $a\in I$, we say $f$ is differentiable.
\end{definition}
Geometrically $f^\prime(a)$ is the slope of the tangent line of $f$ at $a$.\\
Another (formal) way to put this is to observe that the definition is essentially $\forall \epsilon>0,\exists\delta>0$,
$$|x-a|<\delta\implies\left|\frac{f(x)-f(a)}{x-a}-c\right|<\epsilon\implies |f(x)-f(a)-c(x-a)|<\epsilon|x-a|$$
Another way we can write is
$$f^\prime(a)=\lim_{h\to 0}\frac{f(a+h)-f(a)}{h}$$
where $h\in I-a$.
\begin{example}
    1. The constant function is differentiable everywhere and has zero derivative everywhere.\\
    2. The identity function is differentiable everywhere and has derivative $1$.\\
    3. $f:(0,\infty)\to\mathbb R$ by $x\mapsto 1/x$ is differentiable and has derivative $f^\prime(a)=-a^{-2}$.\\
    4. (non-example) The function $x\mapsto |x|$ is not differentiable at $0$.
\end{example}
\begin{lemma}
    Suppose $f$ is differentiable at $a$ with derivative $f^\prime(a)$ iff
    $$f(a+h)=f(a)+f^\prime(a)h+\alpha(h)h$$
    where $\alpha(0)=0$ and $\alpha$ is continuous at $0$.
\end{lemma}
Note that given this formula, then $\alpha(h)$ is completely deterined by $f$ and $f^\prime(a)$.
\begin{proof}
    Obvious.
\end{proof}
\begin{corollary}
    $f$ is differentiable at $a$ implies $f$ is continuous at $a$.
\end{corollary}
\begin{proof}
    Follows directly.
\end{proof}
\begin{proposition}
    If $f,g:I\to\mathbb R$ are differentiable at some $a\in I$, so are $f+g,fg$ with derivatives $f^\prime(a)+g^\prime(a),f^\prime(a)g(a)+f(a)g^\prime(a)$ respectively.
\end{proposition}
\begin{proof}
    Trivial.
\end{proof}
\begin{theorem}[Chain Rule]
    Suppose $f:I_1\to I_2$ is differentiable at $a\in I_1$ and $g:I_2\to\mathbb R$ is differentiable at $f(a)$, then $(g\circ f)^\prime(a)=g^\prime(f(a))f^\prime(a)$.
\end{theorem}
\begin{proof}
    Let $b=f(a)$, then by our conditions,
    $$f(a+h)=f(a)+f^\prime(a)h+\alpha(h)h,g(b+k)=g(b)+g^\prime(b)k+\beta(k)k$$
    And $\alpha(0)=\beta(0)=0$ and they are both continuous at $0$.
    Take $k=f(a+h)-f(a)=f^\prime(a)h+\alpha(h)h$, so
    \begin{align*}
        (g\circ f)(a)&=g(b+k)\\
        &=g(b)+g^\prime(b)k+\beta(k)k\\
        &=g(b)+g^\prime(b)(f^\prime(a)h+\alpha(h)h)+\beta(f^\prime(a)h+\alpha(h)h)(f^\prime(a)h+\alpha(h)h)\\
        &=(g\circ f)(a)+g^\prime(f(a))f^\prime(a)h+\phi(h)h
    \end{align*}
    Where $\phi(h)=g^\prime(b)\alpha(h)+\beta(f^\prime(a)h+\alpha(h)h)(f^\prime(a)+\alpha(h))$ which is $0$ at $0$ and is continuous at $0$.
    The chain rule follows.
\end{proof}
\begin{corollary}
    Suppose $f,g:I\to\mathbb R$ are differentiable at $a$ and $g(a)\neq 0$, then $f/g$ is differentiable at $a$ and $(f/g)^\prime=(f^\prime g-fg^\prime)/g^2$.
\end{corollary}
\begin{proof}
    Immediate from the chain rule.
\end{proof}
\begin{corollary}
    The polynomial $p(x)=a_0+a_1x+\cdots+a_nx^n$ is differentiable at all $x\in\mathbb R$, and if $p,q$ are polynomials, $p/q$ is differentiable at $a$ when $q(a)\neq 0$.
\end{corollary}
\begin{proof}
    Trivial.
\end{proof}
\subsection{Mean Value Theorem}
Suppose we have some $f:I\to\mathbb R$.
\begin{definition}
    We say some $c\in I$ is a global maximum of $f$ if $\forall x\in I,f(x)\le f(c)$ and a global minimum if $\forall x\in I,f(x)\ge f(c)$.
\end{definition}
We know that if $I$ is a closed bounded interval, then $f$ has global maxima and minima.
\begin{definition}
    Some $c\in I$ is called a local maximum of $f$ if $\exists\epsilon>0,\forall x\in (c-\epsilon,c+\epsilon)\cap I,f(x)\le f(c)$.
    Similarly it is a local minimum if $\exists\epsilon>0,\forall x\in (c-\epsilon,c+\epsilon)\cap I,f(x)\ge f(c)$.
\end{definition}
\begin{definition}
    $c\in I$ is called an interior point of $I$ if $\exists \epsilon>0,(c-\epsilon,c+\epsilon)\subset I$.
\end{definition}
\begin{proposition}
    Suppose $f:I\to\mathbb R$ and $c$ is an interior point of $I$ where $f$ attains local maximum or local minimum at $c$ and $f$ is differentiable at $c$, then $f^\prime(c)=0$.
\end{proposition}
\begin{proof}
    Suffice to prove the local maximum case.
    Since $c$ is an interior point of $I$, we can choose $\epsilon'>0,(c-\epsilon',c+\epsilon')\subset I$.
    As $c$ is a local maximum, then there is some $\epsilon$ such that $0<\epsilon\le\epsilon'$ and that $\forall x\in (c-\epsilon,c+\epsilon),f(x)\le f(c)$.
    We can easily find sequences $(x_n^-)\in (c-\epsilon,c),(x_n^+)\in (c,c+\epsilon)$ both converging to $c$.
    Let $g:(c-\epsilon,c+\epsilon)\setminus\{c\}\to\mathbb R$ defined by $g(x)=(f(x)-f(c))/(x-c)$, then
    \begin{align*}
        0&\le\lim_{n\to\infty}\frac{f(x_n^-)-f(c)}{x_n^--c}\\
        &=\lim_{x\to c^-}g(x)=\lim_{x\to c}g(x)=\lim_{x\to c^+}g(x)\\
        &=\lim_{n\to\infty}\frac{f(x_n^+)-f(c)}{x_n^+-c} \le 0
    \end{align*}
    Hence $f^\prime(c)=\lim_{x\to c}g(x)=0$.
\end{proof}
\begin{theorem}[Rolle's Theorem]
    Suppose $f:[a,b]\to\mathbb R$ is continuous and is differentiable in $(a,b)$ and $f(a)=f(b)=0$, then there is some $c\in (a,b)$, $f^\prime(c)=0$.
\end{theorem}
\begin{proof}
    It is obvious if $f\equiv 0$, otherwise WLOG $f$ attains some positive value.
    Then let $c$ be a global maximum of $f$, then $c$ is a local maximum and an interior point of $f$ (since $f$ is zero on endpoints but we have assumed that $f$ attains some positive value), then $c\in (a,b)$ and we have $f^\prime(c)=0$ by the preceding proposition.
\end{proof}
\begin{theorem}[Mean Value Theorem]
    If $f:[a,b]\to\mathbb R$ is continuous on $[a,b]$ and differentiable on $(a,b)$, then there is some $c\in (a,b)$ with $f^\prime(c)=(f(b)-f(a))/(b-a)$.
\end{theorem}
\begin{proof}
    Let
    $$p(x)=\frac{1}{b-a}((x-a)f(b)-(x-b)f(a))$$
    which although looks dreadful is just a linear polynomial.
    Note that $p(a)=f(a),p(b)=f(b)$, so it is just a line joining the endpoints.
    Then $g(x)=f(x)-p(x)$ is continuous on $[a,b]$ and differentiable on $(a,b)$, so by Rolle's Theorem, there is some $c\in (a,b)$ sucb that
    $$0=g^\prime(c)=f^\prime(c)-\frac{f(b)-f(a)}{b-a}\implies f^\prime(c)=\frac{f(b)-f(a)}{b-a}$$
    As desired.
\end{proof}
\begin{proposition}
    Suppose $f:[a,b]\to\mathbb R$ is continuous and differentiable in $(a,b)$.\\
    1. $f^\prime\equiv 0\iff f$ is constant.\\
    2. $f^\prime\ge 0\iff f$ is increasing.\\
    3. $f^\prime>0\implies f$ is strictly increasing.
\end{proposition}
\begin{proof}
    Immediate from Mean Value Theorem.
\end{proof}
Note that the converse of the third statement is false in general.
E.g, $f(x)=x^3$.\\
Note that all these depends on the least upper bound property since we used the Mean Value Theorem which depends on Rolle's Theorem which depends on maximum value theorem which depends on Bolzano-Weierstrass Theorem which depends on the Monotone Sequence Theorem which depends on the least-upper-bound property.
\subsubsection{Inverse Function Theorem}
\begin{lemma}
    If $f:[a,b]\to\mathbb R$ satisfies $f^\prime(x)>0$ for all $x\in [a,b]$, then $f:[a,b]\to[f(a),f(b)]$ is a bijection.
\end{lemma}
\begin{proof}
    Injectivity is obvious by Mean Value Theorem.
    For surjectivity we can just exploit Intermediate Value Theorem.
\end{proof}
\begin{lemma}
    Let $I$ be an interval.
    Suppose $f:I\to f(I)$ is a continuous bijection, then $f^{-1}:f(I)\to I$ is also continuous.
\end{lemma}
\begin{proof}
    Just check definition.
\end{proof}
\begin{theorem}[Inverse Function Theorem]
    Suppose $I\subset R$ is an interval and $f^\prime(x)>0$ for any $x\in I$, then the continuous inverse (exists by the preceding lemmas) $f^{-1}:f(I)\to I$ is differentiable and $(f^{-1})^\prime(y)=1/f^\prime(f^{-1}(y))$.
\end{theorem}
\begin{example}
    1. $f:x\mapsto x^n$ on $(0,\infty)$ satisfies the conditions, so its inverse $f^{-1}:y\mapsto y^{1/n}$ has $(f^{-1})^\prime(y)=(1/n)y^{1/n-1}$.\\
    2. Consider $\tan:(-\pi/2,\pi/2)\to\mathbb R$, it shall be invertible with differentiable inverse $\tan^{-1}$, then we can calculate, by the formula, that $(\tan^{-1})(y)=1/(y^2+1)$
\end{example}
\begin{proof}[Proof assuming differentiability]
    Assuming $f^{-1}$ is differentiable, then we know that $f(f^{-1}(y))=y$, then by the chain rule, we get $f^\prime(f^{-1}(y))(f^{-1})^\prime(y)=1$, rearrange to give the formula.
\end{proof}
\begin{proof}[Actual proof]
    Fix $b\in f(I)$ and let $a=f^{-1}(b)$.
    Suppose $(y_n)$ is a sequence in $f(I)\setminus \{b\}$ with $y_n\to b$, and let $x_n=f^{-1}(y_n)$.
    We know that $f$ is differentiable hence continuous, so $f^{-1}$ is continuous by the preceding lemma.
    Therefore $x_n=f^{-1}(y_n)\to f^{-1}(b)=a$.
    Since we know that
    $$\lim_{x\to a}\frac{f(x)-f(a)}{x-a}=f^\prime(a)$$
    We have
    $$\frac{f(x_n)-f(a)}{x_n-a}\to f^\prime(a)$$
    By hypothesis, $f^\prime(a)\neq 0$, so
    $$\frac{f^{-1}(y_n)-f^{-1}(b)}{y_n-b}=\frac{x_n-a}{f(x_n)-f(a)}\to\frac{1}{f^\prime(a)}=\frac{1}{f^\prime(f^{-1}(b))}$$
    which is true for every sequence $y_n\to b$, so the theorem is proved.
\end{proof}
\subsubsection{L'H\^opital's Rule}
\begin{proposition}[L'H\^opital's Rule]
    Suppose $f,g:I\to\mathbb R$ are differentiable with $f(a)=g(a)=0$, and $\exists r>0, g(x)\neq 0\neq g^\prime(x)$ for $0<|x-a|<r$.
    Then if $\lim_{x\to a}f^\prime(a)/g^\prime(a)$ exists and equals $k$, then
    $$\lim_{x\to a}\frac{f(x)}{g(x)}=k$$
\end{proposition}
\begin{proof}
    Suppose $x\in I$ and $x>a$.
    Consider $h(t)=f(t)g(x)-f(x)g(t),t\in [a,x]$.
    Since $f,g$ are differentiable and we have $h(a)=h(a)=0$, so by Rolle's Theorem, there is some $c\in (a,x)$ with $f^\prime(c)g(x)-f(x)g^\prime(c)=h^\prime(c)=0$.
    Therefore
    $$\frac{f^\prime(c)}{g^\prime(c)}=\frac{f(x)}{g(x)}$$
    Same for $x<a$.\\
    Given $\epsilon>0$, choose $\delta>0$ such that
    $$0<|c-a|<\delta\implies\left|\frac{f^\prime(c)}{g^\prime(c)}-k\right|<\epsilon$$
    So when $0<|x-a|<\delta$, one can find $c$ such that $0<|c-a|<|x-a|<\delta$ and $f^\prime(c)/g^\prime(c)=f(x)/g(x)$, which means that
    $$\left|\frac{f(x)}{g(x)}-k\right|=\left|\frac{f^\prime(c)}{g^\prime(c)}-k\right|<\epsilon$$
    which is as we wanted.
\end{proof}
\subsection{Higher Derivatives and Taylor Series}
\begin{definition}
    We say (inductively) that $f:I\to\mathbb R$ is $k$ times differentiable with $k^{th}$ derivative $f^{(k)}$ if $f$ is $(k-1)$ times differentiable with differentiable $(k-1)^{th}$ derivative $f^{(k-1)}$ and $f^{(k-1)})^\prime=f^{(k)}$.
\end{definition}
\begin{example}
    A polynomial is $k$ times differentiable for any $k\in\mathbb N$.
\end{example}
\begin{definition}
    We say $f$ is $C^k$ if $f$ is $k$ times differentiable and $f^{(k)}$ is continuous.
\end{definition}
\begin{example}[Non-example]
    Consider
    $$f(x)=\begin{cases}
        x^2\sin(1/x)\text{, for $x\neq 0$}\\
        0\text{, for $x=0$}
    \end{cases}$$
    So one can verify that $f$ is differentiable but the derivative, namely,
    $$f^\prime(x)=\begin{cases}
        2x\sin(1/x)-\cos(1/x)\text{, for $x\neq 0$}\\
        0\text{, for $x=0$}
    \end{cases}$$
    is not continuous at $0$.
\end{example}
Now, given $f:I\to\mathbb R$, we want a degree $k$ polynomial that best approximates $f$.
The answer depends on the meaning of ``best''.
We can either calculate the total squared difference (we have not defined it and will not care about it), but on the other hand, the theory of differentiation hints that
$$f(x)=f(a)+f^\prime(a)(x-a)=P_1(x)$$
is the best linear approximation of $f$ around $a$.
Why ``around $a$''?
say for example we want to approximate $\sin x$ over a large interval, then the best approximation intuitively might be a line with slope very close to $0$, which is obviously not the tangent at, e.g., $0$.
And we are only interested in the best approximation close to $a$.\\
Now back to our best linear approximation.
The reason we say it is the ``best'' in the sense that $f(x)-P_1(h)=\alpha(h)h$ with $\alpha(0)=0$ and $\alpha$ is continuous at $0$.
We can also easily know that such a $P_1$ is unique so as for such an $\alpha$ to exist by working out their coefficients by substitution and differentiating (with appropriate comments on differentiability, of course).
More generally, for a sufficiently smooth $f$, we can try to approximate $f(a+h)=P_k(a+h)+h^k\alpha(h)$ where $\alpha(0)=0$ and $\alpha$ is continuous at $0$ and $P_k$ is a polynomial.
Using the same trick, such an $P_k$ is unique.
In particular, we obtain $P_k^{(r)}=f^{(r)}$ for any $r\le k$.
\begin{lemma}
    If $a,c_0,c_1,\ldots,c_k\in\mathbb R$, then there is a unique degree $k$ polynomial $P$ with $P^{(i)}(a)=c_i$.
\end{lemma}
\begin{proof}
    Write out $P(x)=a_0+a_1(x-a)+\cdots+a_k(x-a)^k$ and differentiating gives $i!a_i=c_i=P^{(i)}(a)$, so the polynomial is uniquely determined.
\end{proof}
\begin{definition}
    Given $f:I\to\mathbb R$ which is $k$ times differentiable and $a$ is an interior point of $I$, then the $k^{th}$ Taylor polynomial of $f$ centered at $a$ is
    $$P_k(x)=\sum_{i=0}^k\frac{f^{(i)}(a)}{i!}(x-a)^i$$
    which is chosen such that $P_k^{(i)}(a)=f^{(i)}(a)$.
\end{definition}
\begin{remark}
    $P_k(x)$ is the unique polynomial with degree at most $k$ having the property $P_k^{(i)}(a)=f^{(i)}(a)$ by the preceding lemma.
\end{remark}
\begin{theorem}[Taylor's Theorem with Remainder]
    Suppose $f:I\to\mathbb R$ is $(k+1)$ times differentiable, and $[a,x]\subset I$, then
    $$f(x)=\left(\sum_{i=0}^k\frac{f^{(i)}(a)}{i!}(x-a)^i\right)+\frac{f^{(k+1)}(c)}{(k+1)!}(x-a)^{k+1}$$
    for some $c\in (a,x)$.
\end{theorem}
For $k=0$, this reduces to the mean value theorem.
\begin{proof}
    Fix $x>a$ and define
    $$g(t)=f(t)+\sum_{i=1}^k\frac{f^{(i)}(t)}{i!}(x-t)^i+\alpha\frac{(x-t)^{k+1}}{(k+1)!}$$
    where $\alpha$ is chosen such that $g(a)=f(x)$ (which exists as a solution to a nondegenerate linear equation).
    Our goal is to find $c\in (a,x)$ such that $f^{(k+1)}(c)=\alpha$.
    We have, by definition,
    $$f(x)=g(a)=f(a)+\sum_{i=1}^k\frac{f^{(i)}(a)}{i!}(x-a)^i+\alpha\frac{(x-a)^{k+1}}{(k+1)!}$$
    Now $f$ is $k+1$ times differentiable so $g$ is differentiable and $g(a)=f(x)=g(x)$, so by Rolle's Theorem, there is some $c\in (a,x)$ such that $g^\prime(c)=0$.
    But we have
    $$g^\prime(t)=f^\prime(t)+\sum_{i=1}^k\left(\frac{f^{(i+1)}(t)}{i!}(x-t)^i-\frac{f^{(i)}(t)}{(i-1)!}(x-t)^{i-1}\right)-\alpha\frac{(x-t)^k}{k!}$$
    The first part of this expression is telescoping, so we can simplify it to
    $$g^\prime(t)=(f^{(k+1)}(t)-\alpha)\frac{(x-t)^k}{k!}$$
    Now plugging in $t=c$ shows $f^{(k+1)}(c)=\alpha$, which completes the proof.
\end{proof}
If $f$ is infinitely differentiable, one can then form the Taylor series
$$\sum_{i=0}^\infty\frac{f^{(i)}(a)}{i!}(x-a)^i$$
as the formal limit of the Taylor polynomilas $P_k(x)$, which sometimes does converge and equals $f$.
If we understand the derivatives of $f$ well-enough and that those derivatives do behave nicely, then it is sometimes possible to use Taylor's Theorem to prove that
$$p(x)=\lim_{n\to\infty}P_n(x)=f(x)$$
\begin{example}
    Take $f(x)=\cos x$ (which, albeit hasn't been properly defined, we shall assume standard facts about), then the Taylor series about $0$ is
    $$1-\frac{x^2}{2!}+\frac{x^4}{4!}-\frac{x^6}{6!}+\cdots=\sum_{n=0}^\infty(-1)^n\frac{x^{2n}}{(2n)!}$$
    The series always converges by the ratio test.
    Must the series converges to $\cos x$ then?
    The $(2k)^{th}$ partial sum of the series is exactly $P_{2k}(x)$ of $\cos$ about $0$, then there is $c\in (0,x)$ with
    $$|f(x)-P_{2k}(x)|=\left|\frac{f^{2k+1}(c)}{(2k+1)!}x^{2k+1}\right|\le\frac{|x|^{2k+1}}{(2k+1)!}\to 0$$
    as $k\to\infty$, so the series does converge to $\cos x$.
\end{example}
\begin{definition}
    We say $f:I\to\mathbb R$ is $C^\infty$ (or smooth) if $f$ is $C^k$ for all $k\in\mathbb N$.
\end{definition}
So for a smooth $f:I\to\mathbb R$, one can form the Taylor series (formally)
$$\sum_{n=0}^\infty\frac{f^{(n)}(a)}{n!}(x-a)^n$$
But even if the series converges for all $x\in I$, it may not converge to $f(x)$ for $x\neq a$.
Philosophically, one cannot fully expect that the behaviour of $f$ near $a$ control its global behaviour.
\footnote{For complex-valued functions, however, the local behaviour of a nice enough function $f$ can be controlled by its local properties.}
Indeed, for $f_1,f_2:\mathbb R\to\mathbb R$ smooth and $a<b$, it is possible to construct some smooth $f:\mathbb R\to\mathbb R$ such that $f|_{<a}=f_1|_{<a}$ and $f|_{>b}=f_2|_{>b}$.
This will be proven in example sheet.
\subsection{Complex Differentiation}
\begin{definition}
    If $c\in\mathbb C,r\in[0,\infty)$, we define the open ball of radius $r$ centered at $c$ by $B_r(c)=\{z\in\mathbb C:|z-c|<r\}$.
\end{definition}
\begin{definition}
    A subset $\Omega\subset\mathbb C$ is open if $\forall c\in\Omega,\exists\epsilon>0,B_\epsilon(c)\subset\Omega$.
\end{definition}
\begin{example}
    1. $\mathbb C,\varnothing$ are open.\\
    2. Open balls are open.\\
    3. (non-example) $\{z\in\mathbb C:|z-a|\le r\}$ is not open, but its complement is.
\end{example}
\begin{definition}
    Suppose $\Omega\in\mathbb C$ is open and $c\in\Omega$.
    For a function $f:\Omega\to\mathbb C$, we say $f$ is complex differentiable at $c$ if
    $$\lim_{z\to c}\frac{f(z)-f(c)}{z-c}$$
    exists.
    In case it does exist, we write it as $f^\prime(c)$.
\end{definition}
All the rules in real differentiation still applies, but we don't really have mean value theorem.
We can write $f(x+iy)=u(x,y)+iv(x,y)$ where $u,v:\tilde{\Omega}\to\mathbb R$ where $\tilde{\Omega}=\{(x,y)\in\mathbb R^2:x+iy\in\Omega\}$.
\begin{definition}
    Say $\tilde{\Omega}$ is open if $\Omega$ is.
    Then the partial derivatives of $F:\tilde{\Omega}\to\mathbb R$ at $(a,b)\in\tilde{\Omega}$ are
    \begin{align*}
        \left.\frac{\partial F}{\partial x}\right|_{(a,b)}&=\lim_{h\to 0}\frac{F(a+h,b)-F(a,b)}{h}\\
        \left.\frac{\partial F}{\partial y}\right|_{(a,b)}&=\lim_{h\to 0}\frac{F(a,b+h)-F(a,b)}{h}
    \end{align*}
    Provided that they exist.
\end{definition}
Alternatively we can define $g_b=F(x,b)$ and so
$$\left.\frac{\partial F}{\partial x}\right|_{(a,b)}=g^\prime_b(a)$$
Similar for the other component.
\begin{proposition}
    If $f:\Omega\to\mathbb C$ is complex differentiable at $c=a+ib$ with derivative $f^\prime(c)=\alpha+i\beta$, then the partial derivatives of $u,v$ at $(a,b)$ exist and satisfy
    \begin{align*}
        \left.\frac{\partial u}{\partial x}\right|_{(a,b}&= \left.\frac{\partial v}{\partial y}\right|_{(a,b)}=\alpha\\
        -\left.\frac{\partial u}{\partial y}\right|_{(a,b}&=\left.\frac{\partial v}{\partial x}\right|_{(a,b)}=\beta
    \end{align*}
\end{proposition}
\begin{proof}
    For any sequence $h_n\to 0$ in $\mathbb R\setminus\{0\}$, then it is also a sequence in $\mathbb C\setminus\{0\}$, so since $f$ is differentiable at $c$,
    \begin{align*}
        \alpha+i\beta&=f^\prime(c)\\
        &=\lim_{h\to 0}\frac{f(c+h)-f(c)}{h}\\
        &=\lim_{n\to\infty}\frac{u(a+h_n,b)-u(a,b)}{h_n}+i\lim_{n\to\infty}\frac{v(a+h_n,b)-v(a,b)}{h_n}
    \end{align*}
    So $\partial u/\partial x$ and $\partial v/\partial x$ exist and are $\alpha,\beta$ respectively.
    We can get the other half of the equations by replacing $h_n$ by $ih_n$ and casting exactly the same argument.
\end{proof}
\begin{remark}
    If $f:\Omega\to\mathbb C$ is complex differentiable, then (assuming $C^1$ partial derivatives and symmetry of partial derivatives)
    $$\frac{\partial^2u}{\partial x^2}+\frac{\partial^2u}{\partial y^2}=0$$
    which is the Laplace Equation.
\end{remark}
\begin{proposition}
    Suppose $F:B_r((a,b))\to\mathbb R$ has partial derivatives with
    $$\frac{\partial F}{\partial x}=\frac{\partial F}{\partial y}=0$$
    then $F$ is constant.
\end{proposition}
\begin{proof}
    For $(a',b')\in B_r((a,b))$, we consider the functions $f_b(x)=F(x,b)$ and $f_a(x)=F(a',x)$ on the reals.
    Then they are differentiable and using mean value theorem on $f_b$ wrt $a,a'$ shows $f(a,b)=f(a',b)$, and using it again on $f_a$ wrt $b,b'$ shows $f(a',b)=f(a',b')$, so $f(a,b)=f(a',b')$.
\end{proof}
\begin{corollary}
    If $f:B_r(c)\to\mathbb C$ has derivative zero, then $f$ is constant.
\end{corollary}
\begin{proof}
    Immediate.
\end{proof}
\begin{remark}
    Complex differentiable functions $f:\mathbb C\to\mathbb C$ behave very nicely.
    Firstly, if $f$ is complex differentiable, so is $f^\prime$, consequently $f$ is $C^\infty$.
    Even better, its Taylor series always converges to itself.
    Secondly, if $f$ is bounded then $f$ is constant.
\end{remark}
    \section{Special Functions}
\subsection{Power Series}
Recall that if $\sum_nc_n$ converges then $c_n\to0$.
We also have a strong convergence test where if $|c_n|\le a_n$ for all $n$ and $\sum_na_n$ converges, then $\sum_nc_n$ converges.
We used it to porve the ratio test: suppose $\lim_{n\to\infty}|c_{n+1}/c_n|$ exists and equals $\alpha$, then $\alpha<1$ implies $\sum_nc_n$ converges and $\alpha>1$ implies it diverges.
\begin{definition}
    A power series centered at $a\in\mathbb C$ is a series that can be written in the form
    $$\sum_{n=0}^\infty c_n(z-a)^n,c_n\in\mathbb C$$
\end{definition}
\begin{example}
    The series
    $$\sum_{n=0}^\infty (z-1)^n,\sum_{n=0}\frac{z^n}{n!},\sum_{n=0}^\infty (-1)^n\frac{z^{2n}}{(2n)!}$$
    are power series.
    In general, any Taylor series around $a$ is a power series centered at $a$.
\end{example}
For a power series and each $z\in\mathbb C$, we can get a seperate series.
Then we want to ask for which $z\in\mathbb C$ does the series converges.
\begin{example}
    $\sum_n(z-1)^n$ converges in $|z-1|<1$ and diverges otherwise.
    On the other hand, $\sum_nz^n/n!,\sum_n(-1)^nz^{2n}/(2n)!$ converge absolutely.
    But $\sum_nn!z^n$ converges iff $z=0$.
\end{example}
Note that suppose $|c_{n+1}/c_n|\to\alpha$, then
$$\lim_{n\to\infty}\left|\frac{c_{n+1}(z-a)^{n+1}}{c_n(z-a)^n}\right|=\alpha|z-a|$$
So for $|z-a|<1/\alpha$, the series converges and $|z-a|>1/\alpha$ means it diverges.
But we do not know about the case $|z-a|=\alpha$.\\
We want to show similar result in general (i.e. $\alpha$ may not exist).
\begin{lemma}[Key Estimate]
    Suppose $p_z$ is the series
    $$\sum_{n=0}^\infty c_n(z-a)^n$$
    and that $p_{z_0}$ converges, the there is $M>0$ such that
    $$c_n\le\frac{M}{|z_0-a|^n}$$
\end{lemma}
\begin{proof}
    We have $|c_n||z_0-a|^n\to 0$, so there is some $N$ such that $|c_n||z_0-a|^n<1$ for $n>N$, so $|c_n|\le 1/|z_0-a|^n$ for $n>N$, but we can take care of the rest by taking maximum since there are only finitely many terms left.
\end{proof}
\begin{theorem}
    Suppose
    $$P(z_0)=\sum_{n=0}^\infty c_n(z-a)^n$$
    converges, then $P(z)$ converges absolutely whenever $|z-a|<|z_0-a|$.
\end{theorem}
\begin{proof}
    Given $z,|z-a|<|z_0-a|$, let $\alpha=(z-a)/(z_0-a)$, so $|\alpha|<1$, so choose $M$ as in the preceding lemma gives $c_n|z-a|^n\le M|\alpha|^n$, but $\sum_mM|\alpha|^n$ converges as a geometric series, so $P(z)$ converges absolutely.
\end{proof}
\begin{corollary}
    With $P(z)$ as above, either $P(z)$ converges for all $z\in\mathbb C$ or there exists some $R>0$ such that $P(z)$ converges absolutely for $|z|<R$ and diverges for $|z|>R$.
\end{corollary}
\begin{proof}
    Immediate.
\end{proof}
Such an $R$ is obviously unique, so we define
\begin{definition}
    Such an $R$ is called the radius of convergence of the power series.
\end{definition}
Note that the series may or may not converge on the circle $|z-a|=R$.
\begin{example}
    The following series have radius of convergence $1$:
    $$\sum_{n=0}^\infty nz^n,\sum_{n=1}^\infty\frac{z^n}{n^2},\sum_{n=0}^\infty\sum_{n=1}^\infty\frac{z^n}{n}$$
    But they exhibit different behaviours on $|z|=1$.
\end{example}
\begin{proposition}
    Suppose $P(z)$ has radius of convergence $R$, then\\
    1. $c_n(z-a)^n\to 0$ for $|z|<R$\\
    2. $c_n(z-a)^n$ is unbounded for $|z|>R$.
\end{proposition}
\begin{proof}
    Trivial.
\end{proof}
What we really want to do is to differentiate them.
Suppose we have
$$P(z)=\sum_{n=0}^\infty c_n(z-a)^n$$
Then let $R>0$ be its radius of convergence, then $P:B_R(a)\to\mathbb C$ is a function.
If $P$ is differentiable, we obviously want it to be
$$Q(z)=\sum_{n=1}^\infty nc_n(z-a)^{n-1}$$
which is also a power series.
So we will want $Q$ to have radius of convergence $R$, $P$ to be differentiable and $P^\prime=Q$.
\begin{proposition}
    $Q$ has radius of convergence $R$.
\end{proposition}
\begin{proof}
    Suppose $P(z_0)$ converges with $z_0\neq a$, then there is some $M>0$ with $|c_n|\le M/|z_0-a|^n$ for all $n$.
    Then for any $z$ with $|z-a|<|z_0-a|$,
    $$|nc_n(z-a)^{n-1}|\le nC\alpha^{n-1},C=M/|z_0-a|,\alpha=|z-a|/|z_0-a|<1$$
    The rest is trivial.
\end{proof}
Since we can shift stuff around, we assume $a=0$ for the moment.
The preceding proposition then says
$$P(z)=\sum_{n=0}^\infty c_nz^n,Q(z)=\sum_{n=1}^\infty nc_nz^{n-1}$$
both have radius of convergence $R$.
What we want to show now is that $P$ is complex differentiable on $B_R(0)$ and its derivative is $Q$.
The optimal way to do this is to use the idea of uniform convergence, but we do not have that tool right now, so here goes the clumsy solution.
For $z\in B_R(0)$, consider
\begin{align*}
    P(z+h)-P(z)-hQ(z)&=\sum_{n=0}^\infty c_n(z+h)^n-\sum_{n=0}^\infty c_nz^n-h\sum_{n=1}^\infty nc_nz^{n-1}\\
    &=\sum_{n=0}^\infty c_n((z+h)^n-z^n-nhz^{n-1})\\
    &=\sum_{n=0}^\infty c_n\sum_{k=2}^n\binom{n}{k}h^kz^{n-k}\\
    &=\sum_{n=0}^\infty h^2c_n\sum_{k=2}^n\binom{n}{k}h^{k-2}z^{n-k}\\
    &=\sum_{n=0}^\infty h^2c_nr_n(h;z)
\end{align*}
But $r_n(h;z)$ is just a polynomial.
So intuitively we want to show that its growth is slow, so we claim
\begin{lemma}
    For any $z,h\in\mathbb C$,
    $$|r_n(h;z)|\le\binom{n}{2}(|h|+|z|)^{n-2}$$
\end{lemma}
\begin{proof}
    First suppose that $z,h\in\mathbb R_{\ge 0}$, then by using Taylor's Theorem on $f(z)=z^n$, we have
    $$f(z+h)=f(z)+hf^\prime(z)+\frac{h^2}{2}f^{\prime\prime}(y)$$
    for some $y\in [z,z+h]$, so
    $$r_n(h,z)=\binom{n}{2}y^{n-2}\le\binom{n}{2}(z+h)^{n-2}$$
    For the general case,
    \begin{align*}
        |r_n(h,z)|&=\left|\sum_{i=0}^n\binom{n}{2}h^{i-2}z^{n-i}\right|\\
        &\le\sum_{i=0}^n\binom{n}{2}|h|^{i-2}|z|^{n-i}\\
        &=r_n(|h|,|z|)\le\binom{n}{2}(|z|+|h|)^{n-2}
    \end{align*}
    by above.
\end{proof}
\begin{theorem}\label{tbt_diff}
    $P$ is complex differentiable on $B_R(a)$ and its derivative is $Q$.
\end{theorem}
\begin{proof}
    Again take WLOG $a=0$.
    Let $0<\rho<R-r_1=R-|z|$ and let $r_2=r_1+\rho<R$, then for $h\in B_\rho(0)$, $z+h\in B_\rho(z)\subset B_{r_2}(0)\subset B_R(0)$.
    Pick $r_3$ such that $r_2<r_3<R$, then $P(r_3)$ converges, so $\exists M>0$ such that $|c_n||r_3|^n\le M$ for any $n$.
    Then
    \begin{align*}
        |c_nr_n(h;z)|&\le\frac{M}{r_3^n}\binom{n}{2}(|h|+|z|)^{n-2}\\
        &\le\frac{M}{r_2^2}\binom{n}{2}\alpha^n,\alpha=\frac{r_2}{r_3}<1
    \end{align*}
    Now the series $\sum_n\binom{n}{2}\alpha^n$ converges due to the ratio test, consequently
    $$\sum_{n=0}^\infty c_nr_n(h;z)\to w$$
    For some complex number, which we assume has modulus at most $C>0$, then
    \begin{align*}
        P(z+h)-P(z)-hQ(z)=h^2\sum_{n=0}^\infty c_nr_n(h;z)=h^2\beta(h)
    \end{align*}
    Where $\beta(h)$ is the series sum.
    Let $\alpha(h)=h\beta(h)$, then $|\alpha(h)|\le C|h|$, so $\alpha(h)\to 0$ as $h\to 0$.
    But also $\alpha(0)=0$, so $\alpha$ is $0$ at $0$ and continuous at $0$ and we have $P(z+h)=P(z)+hQ(z)+h\alpha(h)$, so the proof is done.
\end{proof}
\begin{corollary}
    A power series
    $$P(z)=\sum_{n=0}^\infty c_n(z-a)^n$$
    with radius of convergence $R>0$ is infinitely differentiable on $B_R(a)$ and $n!c_n=f^{(n)}(a)$.
\end{corollary}
\begin{proof}
    Immediate.
\end{proof}
\begin{example}
    Consider the sum $\sum_nn^2/2^n$.
    We have $\sum_nz^n=1/(1-z)$ for $|z|<1$, so by differentiating and building stuff we have $\sum_nn^2z^n=z(1+z)/(1-z)^3$, so the series converges to $6$ by taking $z=1/2$.
\end{example}
\begin{remark}
    In general, it is false to say $f_n(z)\to f(z)$ implying $f_n^\prime(z)\to f^\prime(z)$.
    For example one can take $f_n(x)=\sin(nx)/n$ converges to $0$ (even uniformly), but $f_n^\prime(x)=\cos(nx)$ certainly does not converge to $0$.
\end{remark}
\subsection{The Exponentials, etc.}
\begin{definition}
    For $z\in\mathbb C$, the exponential function $\exp(z)$ is defined by
    $$\exp(z)=\sum_{n=0}^\infty\frac{z^n}{n!}$$
    which converges for any $z\in\mathbb C$ by the ratio test.
\end{definition}
We have $\exp^\prime=\exp$ by Theorem \ref{tbt_diff}.
Also $\exp(0)=1$.
\begin{proposition}
    $\forall z,w\in\mathbb C,\exp(z+w)=\exp(z)\exp(w)$.
\end{proposition}
\begin{proof}
    Fix $c\in\mathbb C$ and consider $f(x)=\exp(x)\exp(c-x)$, then by differentiation $f^\prime=0$, so $f$ is constant.
    So $\exp(x)\exp(c-x)=\exp(0)\exp(c-0)=\exp(c)$, so taking $x=z,c=z+w$ gives the result.
\end{proof}
In particular, $\exp(z)\exp(-z)=1$, so $\exp(z)$ is never zero.
And $\exp(-z)=1/\exp(z)$.
By induction, we can easily see that $\exp(nz)=\exp(z)^n$ for all $n\in\mathbb Z$.\\
For $x\in\mathbb R$, $\exp x$ is real since every term of it is real.
\begin{lemma}
    $\exp(\mathbb R)=(0,\infty)$ and it is a strictly increasing bijection $\mathbb R\to (0,\infty)$.
\end{lemma}
\begin{proof}
    Obvious.
\end{proof}
The lemma guarantees a function $\log:(0,\infty)\to\mathbb R$ as the inverse of $\exp:\mathbb R\to (0,\infty)$.
So we have $\log(1)=0,\log(ab)=\log(a)+\log(b)$ for $a,b>0$.
As for the derivative, by the inverse function theorem, $\log^\prime(x)=1/x$.
\begin{proposition}
    $$-\log(1-y)=\sum_{n=1}^\infty\frac{y^n}{n}$$
    for $|y|<1$.
\end{proposition}
\begin{proof}
    Let $p(y)$ be the series in the right hand side, then $p$ does converge for $|y|<1$ by the ratio test.
    Also $p^\prime(y)=1/(1-y)$ by differenating term-by-term.
    So $f(y)=p(y)+\log(1-y)$ has zero derivative, so $f$ is constant.
    Also when $y=0$, we have $f(0)=0$, so $f\equiv 0$, hence $-\log(1-y)=p(y)$.
\end{proof}
\begin{definition}
    For $a\in (0,\infty)$ and $x\in\mathbb R$, we define the power $a^x$ is defined by $\exp(x\log a)$.
\end{definition}
It follows instantly that the power is well-defined, differentiable, and has the series
$$a^x=\sum_{n=0}^\infty\frac{(\log a)^n}{n!}x^n$$
Also we have $a^{x+y}=a^xa^y$ and $(a^x)^y=a^{xy}$.
Define the constant $e=\exp(1)$, so $e^x=\exp(x)$.
\begin{definition}
    For $z\in\mathbb C$, we define
    $$\sin z=\frac{1}{2i}(e^z-e^{-z}),\cos z=\frac{1}{2}(e^z+e^{-z})$$
\end{definition}
Then it follows that they are both differentiable and $\sin^\prime=\cos,\cos^\prime=-\sin$
Also they have power series
$$\sin z=\sum_{n=0}^\infty(-1)^n\frac{1}{(2n+1)!}z^{2n+1},\cos z=\sum_{n=0}^\infty(-1)^n\frac{1}{(2n)!}z^{2n}$$
The familiar trigonometric identities all follows from definition.
In particular, $\sin^2z+\cos^2z=1$, so $\cos,\sin$ has range in $[0,1]$ whenever $z\in\mathbb R$.
We also have Euler's formula: $e^{iz}=\cos z+i\sin z$.\\
Now we recall the alternating series test: If $a_{n+1}\ge a_n\ge 0$ and $a_n\to 0$, then $\sum_n(-1)^na_n$ converges.
Furthermore, if we let $s_n$ be the $n^{th}$ partial sum of the alternating series and $s$ be the limit, then
\begin{proposition}
    For any $n\in\mathbb N$, $s_{2n}\ge s\ge s_{2n+1}$.
\end{proposition}
\begin{proof}
    $s_{2n}$ decreases to $s$ and $s_{2n+1}$ increases to $s$, so the proposition follows immediately.
\end{proof}
\begin{lemma}
    $\sin x>0$ for $x\in (0,2]$.
\end{lemma}
\begin{proof}
    For $x\in (0,2]$, $\sin x$ is an alternating series, so $\sin x\ge x-x^3/6=x(1-x^2/6)>0$.
\end{proof}
\begin{lemma}
    $\cos x>0$ for $x\in [0,1]$ and there is a unique $\alpha\in (1,2]$ with $\cos\alpha=0$.
\end{lemma}
\begin{proof}
    For $x\in [0,1]$, $\cos$ is an alternating series, so $\cos x\ge 1-x^2/2\ge 1/2>0$ for $x\in [0,1]$.\\
    Now the series $\cos 2$ is alternating from the third term, so $\cos 2\le 1-2^2/2+2^4/24=-1/3<0$.
    So there is some $\alpha\in (1,2)$ such that $\cos\alpha=0$ by the intermediate value theorem.
    To see uniqueness, observe that $\cos^\prime=-\sin<0$ in $[1,2]$ by the preceding lemma, so $\cos$ must be strictly monotone hence injective in $[1,2]$, therefore $\alpha$ is unique.
\end{proof}
\begin{definition}
    Let $\alpha$ be as in the preceding lemma, then we define $\pi=2\alpha$.
\end{definition}
Consequenly $\sin\alpha=1$ by $\sin^2z+\cos^2z=1$, and $\cos(x+\alpha)=-\sin x,\sin(x+\alpha)=\cos x$.
So $\sin,\cos$ are periodic with period $2\pi$, and we know the trends in between.
    \section{Integration}
\subsection{The Riemann Integral}
Our goal is the following: for a ``sufficiently nice'' function $f:[a,b]\to\mathbb R$ we want to define its integral
$$\int_a^bf(x)\,\mathrm dx$$
We would like to think of this as the area under the graph of $y=f(x)$, which motivates the Riemann integral.
\begin{definition}
    We say $f:[a,b]\in\mathbb R$ is bounded if $\exists M\in\mathbb R,|f(x)|\le M$ for all $x\in [a,b]$.
\end{definition}
\begin{definition}
    A dissection of the interval $[a,b]$ is a finite subset $D\subset [a,b]$ such that $a,b\in D$.
    If $D$ is a dissection, we can write $D=\{a_0,\ldots,a_n\}$ where $a=a_0<a_1<\cdots<a_n=b$.
\end{definition}
Note that if $D,D'$ are both dissections, so is $D\cup D'$.
\begin{definition}
    If $f:[a,b]\to\mathbb R$ is bounded and a dissection $D$ consists of $a=a_0<a_1<\cdots<a_n=b$, then we define the upper and lower Riemann sums wrt $D$ by
    $$U(f,D)=\sum_{i=1}^n(a_i-a_{i-1})\sup_{a_{i-1}<x<a_i}f(x),L(f,D)=\sum_{i=1}^n(a_i-a_{i-1})\inf_{a_{i-1}<x<a_i}f(x)$$
    respectively.
    They exists as $f$ is bounded.
\end{definition}
Since $f$ is bounded, $U,L$ are bounded over all dissections $D$.
\begin{lemma}
    If $D\subset D'$ are dissections of $[a,b]$, then $U(f,D)\ge U(f,D')$ and $L(f,D)\le L(f,D')$.
\end{lemma}
\begin{proof}
    Obvious.
\end{proof}
\begin{lemma}
    If $D_1,D_2$ are dissections of $[a,b]$, then $U(f,D_1)\ge L(f,D_2)$.
\end{lemma}
\begin{proof}
    Trivial.
\end{proof}
Consider the nonempty sets
$$U=\{U(f,D):D\text{ dissects }[a,b]\},L=\{L(f,D):D\text{ dissects }[a,b]\}$$
It then follows by the preceding lemma that $U$ is bounded below and $L$ is bounded above, so
\begin{definition}
    We define
    $$U(f)=\inf U=\inf_{D\text{ dissects }[a,b]}U(f,D),L(f)=\sup L=\sup_{D\text{ dissects }[a,b]}L(f,D)$$
    as the upper and lowe Riemann sums of $f$ over $[a,b]$.
\end{definition}
\begin{lemma}
    $U(f)\ge L(f)$.
\end{lemma}
\begin{proof}
    Follows from definition.
\end{proof}
So we will want to define
\begin{definition}
    If $L(f)=U(f)$, then we say $f$ is (Riemann) integrable on $[a,b]$ and write
    $$\int_a^bf(x)\,\mathrm dx=L(f)=U(f)$$
\end{definition}
\begin{example}
    1. Consider $f(x)=c$ for a constant $c$ for all $x\in [a,b]$, then consider $D=\{a,b\}$, then $U(f,D)=c(b-a)=L(f,D)$, hence $U(f)=c(b-a)=L(f)$, therefore $f$ is Riemann integrable and
    $$\int_a^bf(x)\,\mathrm dx=c(b-a)$$
    2. Take $f:[a,b]\to\mathbb R$ by $f(x)=0$ if $x\in\mathbb Q$ and $f(x)=1$ if $x\in\mathbb R\setminus\mathbb Q$.
    Note that if $a_{i-1}<a_i$, then $[a_{i-1},a_i]$ contains both rational and irrational numbers, so it follows that $U(f,D)=b-a$ and $L(f,D)=0$ for any dissection $D$ of $[a,b]$, therefore $U(f)=b-a\neq 0=L(f)$, hence $f$ is not Riemann integrable.
\end{example}
\subsection{Properties of the Integral}
\begin{proposition}
    $f:[a,b]\to\mathbb R$ is integrable on $[a,b]$ and
    $$\int_a^bf(x)\,\mathrm dx=I$$
    iff for every $\epsilon>0$, there is a dissection $D$ of $[a,b]$ with $U(f,D)< I+\epsilon$ and $L(f,D)>I-\epsilon$
\end{proposition}
\begin{proof}
    Immediate.
\end{proof}
\begin{proposition}
    $f:[a,b]\to\mathbb R$ is integrable on $[a,b]$ iff for any $\epsilon>0$, there is a dissection $D$ of $[a,b]$ such that $U(f,D)-L(f,D)<\epsilon$
\end{proposition}
\begin{proof}
    Consider $\inf_D U(f,D)-L(f,D)$.
\end{proof}
\begin{definition}
    If $D=\{a=a_0<a_1<\ldots<a_n=b\}$ is a dissection, then the mesh of $D$ is the difference $m(D)=\max\{a_i-a_{i-1}\}$.
\end{definition}
Observe that if $\epsilon>0$, then one can find dissection with mesh less than $\epsilon$.
\begin{theorem}
    If $f:[a,b]\to\mathbb R$ is increasing, then $f$ is integrable.
\end{theorem}
\begin{proof}
    Since $f$ is increasing, $f(a_i)\ge f(x)$ whenever $x\le a_i$.
    So
    $$\sup_{a_{i-1}<x<a_i}f(x)\le f(a_i),\inf_{a_{i-1}<x<a_i}f(x)\ge f(a_{i-1})$$
    Hence
    $$U(f,D)=\sum_{i=1}^n(a_i-a_{i-1})\sup_{a_{i-1}<x<a_i}f(x)\le\sum_{i=1}^n(a_i-a_{i-1})f(a_i)$$
    and
    $$L(f,D)=\sum_{i=1}^n(a_i-a_{i-1})\inf_{a_{i-1}<x<a_i}f(x)\ge\sum_{i=1}^n(a_i-a_{i-1})f(a_{i-1})$$
    So by substracting
    $$U(f,D)-L(f,D)\le \sum_{i=1}^n(a_i-a_{i-1})(f(a_i)-f(a_{i-1}))\le m(D)(f(b)-f(a))<\epsilon$$
    If $f(b)=f(a)$ then $f$ is constant and we know that it is integrable.
    Otherwise we take $D$ with $m(D)<\epsilon/(f(b)-f(a))$, then by the preceding lemma, $f$ is integrable on $[a,b]$.
\end{proof}
\begin{proposition}
    Suppose $f,g:[a,b]\in\mathbb R$ are integrable, then\\
    1. If $\lambda\in\mathbb R$, then $f+\lambda g$ is integrable and
    $$\int_a^bf(x)+\lambda g(x)\,\mathrm dx=\int_a^bf(x)\,\mathrm dx+\lambda\int_a^bg(x)\,\mathrm dx$$
    2. For any $c\in [a,b]$, $f|_{[a,c]},f|_{[c,b]}$ are integrable and
    $$\int_a^cf(x)\,\mathrm dx+\int_c^bf(x)\,\mathrm dx=\int_a^bf(x)\,\mathrm dx$$
    3. If $f\ge g$ on $[a,b]$, then
    $$\int_a^bf(x)\,\mathrm dx\ge\int_a^bg(x)\,\mathrm dx$$
    4. $|f|$ is integrable and
    $$\left|\int_a^bf(x)\,\mathrm dx\right|\le\int_a^b|f(x)|\,\mathrm dx$$
\end{proposition}
\begin{proof}
    Trivial but need patience.
\end{proof}
\subsection{Fundamental Theorem of Calculus}
Recall that continuous function on closed interval is bounded.
\begin{theorem}\label{cont_int}
    If $f:[a,b]\to\mathbb R$ is continuous, then $f$ is integrable.
\end{theorem}
\begin{lemma}
    Suppose $U(f,D)-L(f,D)>c>0$, then there is $a_i\in D$ such that $\exists x,y\in (a_i,a_{i+1})$ with $f(x)-f(y)\ge c/(b-a)$.
\end{lemma}
\begin{proof}
    Basically just pigeonhole principle.
\end{proof}
\begin{proof}[Proof of Theorem \ref{cont_int}]
    If $f$ is not integrable, then there is some $\epsilon>0$ such that $U(f,D)-L(f,D)>\epsilon$ for all dissections $D$ of $[a,b]$.
    Let $0<\alpha<\epsilon/(b-a)$.
    By the preceding lemma, if $D$ is a dissection of $[a,b]$, then there is some $a_i\in D$ such that $\exists x,y\in (a_i,a_{i+1}),f(x)-f(y)\ge\alpha$
    Consider the dissections $D_n$ with mesh $m(D_n)<1/n$, then we can find $a_i^{(n)}\in D_n$ and $x_n,y_n\in (a_i^{(n)},a_{i+1}^{(n)}),f(x_n)-f(y_n)\ge\alpha$.
    Since $m(D_n)<1/n$, we know that $|x_n-y_n|<1/n$, so $x_n-y_n\to 0$.
    Now by Bolzano-Weierstrass, we can find a converging subsequence $x_{n_k}\to x$ of $x_n$.
    Also $x\in [a,b]$ since $[a,b]$ is closed.
    Note also that since $x_n-y_n\to 0$, we have $y_{n_k}\to x$.
    But $f$ is continuous, so $0<\alpha\le f(x_{n_k})-f(y_{n_k})\to 0$, contradiction.
\end{proof}
\begin{theorem}[Fundamental Theorem of Calculus, Version 1]
    Suppose $f:[a,b]\to\mathbb R$ is continuous, then define
    $$G(x)=\int_a^xf(t)\,\mathrm dt$$
    for $x\in [a,b]$.
    Then $G$ is differentiable and $G^\prime=f$.
\end{theorem}
Note that given $x\in [a,b]$, since $f$ is continuous in $[a,b]$, then it is continuous and hence integrable in $[a,x]$, so $G(x)$ is always well-defined.
\begin{proof}
    We need to show that, as $h\to 0$,
    $$\frac{G(x+h)-G(x)}{h}\to f(x)$$
    for fixed $x\in[a,b]$.
    Assume for the moment that $h>0$, then
    $$G(x+h)-G(x)=\int_x^{x+h}f(t)\,\mathrm dt$$
    Given $\epsilon>0$, we can find $\delta>0$ such that $|f(t)-f(x)|<\epsilon$ whenever $|t-x|<\delta$.
    So whenever $0<h<\delta$, we have $f(x)-\epsilon<f(t)<f(x)+\epsilon$ for $t\in [x,x+h]$, so
    $$f(x)h-\epsilon h=\int_x^{x+h}f(x)-\epsilon\,\mathrm dt\le\int_{x}^{x+h}f(t)\,\mathrm dt\le\int_x^{x+h}f(x)-\epsilon\,\mathrm dt=f(x)h+\epsilon h$$
    hence
    $$\left|\frac{G(x+h)-G(x)}{h}-f(x)\right|\le\epsilon$$
    for $0<h<\delta$, hence
    $$\lim_{h\to 0^+}\frac{G(x+h)-G(x)}{h}=f(x)$$
    Using exactly the same way, we also have
    $$\lim_{h\to 0^-}\frac{G(x+h)-G(x)}{h}=f(x)$$
    Combining them gives the result.
\end{proof}
\begin{definition}
    If $F,f:[a,b]\to\mathbb R$ with $F^\prime=f$, we say $F$ is an antiderivative of $f$.
\end{definition}
So the theorem says every continuous function on closed interval has an antiderivative.
\begin{theorem}[Fundamental Theorem of Calculus, Version 2]
    If $f:[a,b]\to\mathbb R$ is continuous and $F:[a,b]\to\mathbb R$ is differentiable with $F^\prime=f$, then
    $$\int_a^bf(x)\,\mathrm dx=F(b)-F(a)$$
\end{theorem}
\begin{proof}
    Let
    $$G(x)=\int_a^xf(t)\,\mathrm dt$$
    Then $G$ is differentiable and $G^\prime=f=F^\prime$, so $(G-F)^\prime=0$, hence $G-F$ is constant hence and equals $G(a)-F(a)=-F(a)$.
    Hence
    $$\int_a^bf(x)\,\mathrm dx=G(b)=F(b)+G(b)-F(b)=F(b)-F(a)$$
    Which is what we want.
\end{proof}
Note that we have used here many theorems about the real numbers, like the mean value theorem.
\begin{corollary}
    Suppose $f:[a,b]\to\mathbb R$ is continuous and $g:[c,d]\to\mathbb R$ is $C^1$, and $g(c)=a, g(d)=b$, then
    $$\int_a^bf(t)\,\mathrm dt=\int_c^df(g(s))g^\prime(s)\,\mathrm ds$$
\end{corollary}
\begin{proof}
    Chain rule and FTC.
\end{proof}
\begin{corollary}[Integration by Part]
    Suppose $f,g:[a,b]\to\mathbb R$ are both $C^1$, then
    $$\int_a^bg^\prime(t)h(t)\,\mathrm dt=g(b)h(b)-g(a)h(a)-\int_a^bg(t)h^\prime(t)\,\mathrm dt$$
\end{corollary}
\begin{proof}
    Product rule and FTC.
\end{proof}
Recall that if $f:[a,b]\to\mathbb R$ is $C^k$, the $k^{th}$ Taylor polynomial centered at $a$ is
$$P_k(x)=\sum_{i=0}^k\frac{f^{(i)}(a)}{i!}(x-a)^i$$
\begin{theorem}[Integral Form of Taylor's Theorem]
    If $f:[a,b]\to\mathbb R$ is $C^k$ and $x\in[a,b]$, then $R_k(x)=f(x)-P_k(x)$ has
    $$R_k(x)=\int_a^x\frac{(x-t)^{k-1}}{(k-1)!}f^{(k)}(t)\,\mathrm dt$$
\end{theorem}
\begin{proof}
    Induction by using integration by part.
\end{proof}
\begin{corollary}
    If $f:[a,b]\to\mathbb R$ is $C_k$, then there is a constant $M\in\mathbb R$ such that $|f(x)-P_{k-1}(x)|\le M|x-a|^k$.
\end{corollary}
\begin{proof}
    Since $f$ is $C^k$, $|f^{(k)}|$ is bounded.
    The rest follows instantly.
\end{proof}
If $x>0,s\in\mathbb R$, we defined $x^s=\exp(s\log x)$, so $\mathrm dx^s/\mathrm dx=sx^{s-1}$.
Now consider $f(x)=(1+x)^s$.
It has Taylor polynomials (centered at $0$)
$$P_k(x)=\sum_{i=0}^k\binom{s}{i}x^i,\binom{s}{i}=\frac{1}{i!}\prod_{j=0}^{i-1}(s-j)$$
We write $P(x)$ to be the formal sum when $k\to\infty$.
By the ratio test, $P$ converges whenever $|x|<1$.
\begin{theorem}[Newton's Binomial Theorem]\label{gen_bin}
    For $x\in (-1,1)$, $P(x)$ converges to $(1+x)^s$.
\end{theorem}
\begin{lemma}
    $$k\binom{s}{k}=s\binom{s-1}{k-1},\binom{s}{k-1}+\binom{s}{k}=\binom{s+1}{k}$$ 
\end{lemma}
\begin{proof}
    Trivial.
\end{proof}
\begin{proof}[Proof of Theorem \ref{gen_bin}]
    By termwise differentiation,
    $$(1+x)P^\prime(x)=(1+x)\sum_{i=0}^\infty i\binom{s}{i}x^{i-1}=sP(x)$$
    by the preceding lemma.
    So consider $h(x)=(1+x)^{-s}P(x)$, so $h^\prime(x)=0$ by the differential equation we obtained above, hence $h$ is constant in $(-1,1)$, hence $h\equiv h(0)=1$.
    Therefore $P(x)=(1+x)^s$ for each $x\in (-1,1)$.
\end{proof}
\subsection{Improper Integral}
Our definition of integral
$$\int_a^bf(x)\,\mathrm dx$$
requires $a,b<\infty$ and $f$ is defined and bounded in $[a,b]$.
But we'd also like to think about things like
$$\int_1^\infty\frac{\mathrm dx}{x^2+1},\int_0^1\frac{\mathrm dx}{\sqrt{x}}$$
which are not defined in our previous definition of integral.
Suppose $f:[a,b)\to\mathbb R$ is continuous and $b\in\mathbb R\cup\{\infty\}$.
Then if $x\in [a,b)$, then
$$\int_a^xf(t)\,\mathrm dt$$
is well-defined since $f$ is continuous on $[a,x]$.
\begin{definition}
    If $f:[a,b)\to\mathbb R$ is continuous and
    $$\lim_{x\to b^-}\int_a^xf(t)\,\mathrm dt$$
    exists and equals to some $c\in\mathbb R$, we define
    $$\int_a^bf(t)\,\mathrm dt=c$$
    and say that this integral converges.
    If this limit does not exist, we say the integral diverges.\\
    Similarly, if $f:(a,b]\to\mathbb R$ is continuous and
    $$\lim_{x\to a^+}\int_x^bf(t)\,\mathrm dt$$
    exists and equals to $c$, then we define
    $$\int_a^bf(t)\,\mathrm dt=c$$
    And say that this integral converges.
    Otherwise we say this integral diverges.
\end{definition}
Note that by FTC this definition (whenever the integral converges) does coincide with our original definition of integrals.
\begin{example}
    Consider the integral
    $$\int_1^\infty t^{-s}\,\mathrm dt$$
    Then for $s>1$, then integral converges and equals $1/(s-1)$, and it diverges for $s\le 1$.
    Similarly
    $$\int_0^1 t^{-s}\,\mathrm dt$$
    converges iff $s<1$.
\end{example}
More generally,
\begin{definition}
    If $f:(a,b)\to\mathbb R$ is continuous, then choose $c\in (a,b)$, then if
    $$\int_a^cf(t)\,\mathrm dt,\int_c^bf(t)\,\mathrm dt$$
    both converge, then we say the integral of $f(t)$ over $(a,b)$ converges and
    $$\int_a^bf(t)\,\mathrm dt=\int_a^cf(t)\,\mathrm dt+\int_c^bf(t)\,\mathrm dt$$
\end{definition}
Note that this definition (both the convergence and the value of the integral) does not depend on the choice of $c$.
\begin{example}
    We have
    $$\int_{-\infty}^\infty\frac{\mathrm dx}{1+x^2}=\int_{-\infty}^0\frac{\mathrm dx}{1+x^2}+\int_{0}^\infty\frac{\mathrm dx}{1+x^2}=\pi$$
    But
    $$\int_{-\infty}^\infty t\,\mathrm dt$$
    does not converge since
    $$\int_0^\infty f(t)\,\mathrm dt$$
    does not converge.
    $$\int_0^\infty t^{-s}\,\mathrm dt=\int_0^1 t^{-s}\,\mathrm dt+\int_1^\infty t^{-s}\,\mathrm ds$$
    never converges for any value of $s$.
\end{example}
\begin{proposition}[The Integral Test]
    Suppose $f:[1,\infty)\to[0,\infty)$ is continuous and decreasing, then
    $$\sum_{n=1}^\infty f(n)$$
    converges iff
    $$\int_1^\infty f(t)\,\mathrm dt$$
    converges.
\end{proposition}
\begin{example}
    $\sum_nn^{-s}$ converges iff $s>1$ due to our discussion on the integral of $t^{-s}$ over $[1,\infty)$.
\end{example}
\begin{proof}
    Since $f(x)\ge 0$,
    $$F(x)=\int_1^x f(t)\,\mathrm dt$$
    is increasing.
    Let $A=F([1,\infty))$, then if $A$ is bounded above iff the integral converges.
    Similarly since $F(n)\ge 0$, the sequence $s_n$ of partial sums is increasing, so the series converges iff $B=\{s_n\}$ is bounded above.
    Now let $D_n=\{1,2,\ldots,n\}$ be a dissection of $[1,n]$, then
    $$s_n-f(1)=L(f,D_n)\le F(n)\le U(f,D_n)=s_n-f(n)\le s_n-c$$
    where $c=\inf_x f(x)$ (exists as $f$ is bounded below by $0$) since $f$ is decreasing.
    This implies the claim.
\end{proof}
\begin{remark}
    As we see in the proposition, improper integrals behave like series in a certain sense.
    We also have a comparison test for improper integrals.
    If $f,g:[a,b)\to\mathbb R$ satisfies $|g|\le f$ and
    $$\int_a^bf(t)\,\mathrm dt$$
    converges, then
    $$\int_a^bg(t)\,\mathrm dt$$
    converges.
\end{remark}
Using improper integrals, we can define certain interesting functions.
\begin{definition}
    The gamma function $\Gamma:(0,\infty)\to\mathbb R$ is defined by
    $$\Gamma(s)=\int_0^\infty x^{s-1}e^{-x}\,\mathrm dx$$
\end{definition}
Note that the integral always converges by whatever triviality, so this function is always well-defined.
Note also that by integration by part we know that for $s>1$, we have $\Gamma(s+1)=s\Gamma(s)$, hence $\Gamma(n+1)=n!$ for $n\in\mathbb Z_{\ge 1}$.\\
One of the motivation of defining this gamma function is because of the following (here many informal arguments are used):
Now consider an $n$-dimensional sphere $S^{n-1}=\{v\in\mathbb R^n:|v|=1\}$.
Let $V(S^{n-1})$ be the volume (or area, etc.) of $S^{n-1}$, so for example $V(S_1)=2\pi$.
Observe that
$$\int_{\mathbb R^n}f(|v|)\,\mathrm dv=\int_0^\infty V(S^{n-1})r^{n-1}f(r)\,\mathrm dr$$
Now consider $f(r)=e^{-r^2}$ and let $c_n$ be the integral as above, then
$$c_n=\prod_{i=1}^n\int_\mathbb Re^{-x^2}\,\mathrm dx=(c_1)^n$$
But on the other hand $c_n$ equals to
$$\int_0^\infty V(S^{n-1})r^{n-1}e^{-r^2}\,\mathrm dr=\frac{V(S^{n-1})}{2}\Gamma\left( \frac{n}{2} \right)$$
So
$$V(S^{n-1})=\frac{2c_1^n}{\Gamma(n/2)}$$
We know that $c_1=\sqrt{\pi}$ since we know $V(S_1)=2\pi$, and this gives the formula.
\end{document}